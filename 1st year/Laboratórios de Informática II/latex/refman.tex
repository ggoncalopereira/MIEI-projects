\documentclass[twoside]{book}

% Packages required by doxygen
\usepackage{calc}
\usepackage{doxygen}
\usepackage{graphicx}
\usepackage[utf8]{inputenc}
\usepackage{makeidx}
\usepackage{multicol}
\usepackage{multirow}
\usepackage{textcomp}
\usepackage[table]{xcolor}

% NLS support packages
Portuguese
% Font selection
\usepackage[T1]{fontenc}
\usepackage{mathptmx}
\usepackage[scaled=.90]{helvet}
\usepackage{courier}
\usepackage{amssymb}
\usepackage{sectsty}
\renewcommand{\familydefault}{\sfdefault}
\allsectionsfont{%
  \fontseries{bc}\selectfont%
  \color{darkgray}%
}
\renewcommand{\DoxyLabelFont}{%
  \fontseries{bc}\selectfont%
  \color{darkgray}%
}

% Page & text layout
\usepackage{geometry}
\geometry{%
  a4paper,%
  top=2.5cm,%
  bottom=2.5cm,%
  left=2.5cm,%
  right=2.5cm%
}
\tolerance=750
\hfuzz=15pt
\hbadness=750
\setlength{\emergencystretch}{15pt}
\setlength{\parindent}{0cm}
\setlength{\parskip}{0.2cm}
\makeatletter
\renewcommand{\paragraph}{%
  \@startsection{paragraph}{4}{0ex}{-1.0ex}{1.0ex}{%
    \normalfont\normalsize\bfseries\SS@parafont%
  }%
}
\renewcommand{\subparagraph}{%
  \@startsection{subparagraph}{5}{0ex}{-1.0ex}{1.0ex}{%
    \normalfont\normalsize\bfseries\SS@subparafont%
  }%
}
\makeatother

% Headers & footers
\usepackage{fancyhdr}
\pagestyle{fancyplain}
\fancyhead[LE]{\fancyplain{}{\bfseries\thepage}}
\fancyhead[CE]{\fancyplain{}{}}
\fancyhead[RE]{\fancyplain{}{\bfseries\leftmark}}
\fancyhead[LO]{\fancyplain{}{\bfseries\rightmark}}
\fancyhead[CO]{\fancyplain{}{}}
\fancyhead[RO]{\fancyplain{}{\bfseries\thepage}}
\fancyfoot[LE]{\fancyplain{}{}}
\fancyfoot[CE]{\fancyplain{}{}}
\fancyfoot[RE]{\fancyplain{}{\bfseries\scriptsize Gerado em Terça, 2 de Junho de 2015 12\-:24\-:28 para Trabalho L\-I2 por Doxygen }}
\fancyfoot[LO]{\fancyplain{}{\bfseries\scriptsize Gerado em Terça, 2 de Junho de 2015 12\-:24\-:28 para Trabalho L\-I2 por Doxygen }}
\fancyfoot[CO]{\fancyplain{}{}}
\fancyfoot[RO]{\fancyplain{}{}}
\renewcommand{\footrulewidth}{0.4pt}
\renewcommand{\chaptermark}[1]{%
  \markboth{#1}{}%
}
\renewcommand{\sectionmark}[1]{%
  \markright{\thesection\ #1}%
}

% Indices & bibliography
\usepackage{natbib}
\usepackage[titles]{tocloft}
\setcounter{tocdepth}{3}
\setcounter{secnumdepth}{5}
\makeindex

% Hyperlinks (required, but should be loaded last)
\usepackage{ifpdf}
\ifpdf
  \usepackage[pdftex,pagebackref=true]{hyperref}
\else
  \usepackage[ps2pdf,pagebackref=true]{hyperref}
\fi
\hypersetup{%
  colorlinks=true,%
  linkcolor=blue,%
  citecolor=blue,%
  unicode%
}

% Custom commands
\newcommand{\clearemptydoublepage}{%
  \newpage{\pagestyle{empty}\cleardoublepage}%
}


%===== C O N T E N T S =====

\begin{document}

% Titlepage & ToC
\hypersetup{pageanchor=false}
\pagenumbering{roman}
\begin{titlepage}
\vspace*{7cm}
\begin{center}%
{\Large Trabalho L\-I2 }\\
\vspace*{1cm}
{\large Gerado por Doxygen 1.8.6}\\
\vspace*{0.5cm}
{\small Terça, 2 de Junho de 2015 12:24:28}\\
\end{center}
\end{titlepage}
\clearemptydoublepage
\tableofcontents
\clearemptydoublepage
\pagenumbering{arabic}
\hypersetup{pageanchor=true}

%--- Begin generated contents ---
\chapter{Índice dos componentes}
\section{Lista de componentes}
Lista de classes, estruturas, uniões e interfaces com uma breve descrição\-:\begin{DoxyCompactList}
\item\contentsline{section}{\hyperlink{unionelem_1_1dados}{elem\-::dados} }{\pageref{unionelem_1_1dados}}{}
\item\contentsline{section}{\hyperlink{structdadosp}{dadosp} }{\pageref{structdadosp}}{}
\item\contentsline{section}{\hyperlink{structdadosvh}{dadosvh} }{\pageref{structdadosvh}}{}
\item\contentsline{section}{\hyperlink{structelem}{elem} }{\pageref{structelem}}{}
\item\contentsline{section}{\hyperlink{structest__bn}{est\-\_\-bn} }{\pageref{structest__bn}}{}
\item\contentsline{section}{\hyperlink{structstack}{stack} }{\pageref{structstack}}{}
\end{DoxyCompactList}

\chapter{Índice dos ficheiros}
\section{Lista de ficheiros}
Lista de todos os ficheiros com uma breve descrição\-:\begin{DoxyCompactList}
\item\contentsline{section}{\hyperlink{header_8h}{header.\-h} }{\pageref{header_8h}}{}
\item\contentsline{section}{\hyperlink{interpretador_8c}{interpretador.\-c} }{\pageref{interpretador_8c}}{}
\item\contentsline{section}{\hyperlink{parte1_8c}{parte1.\-c} }{\pageref{parte1_8c}}{}
\item\contentsline{section}{\hyperlink{parte2_8c}{parte2.\-c} }{\pageref{parte2_8c}}{}
\item\contentsline{section}{\hyperlink{parte3_8c}{parte3.\-c} }{\pageref{parte3_8c}}{}
\end{DoxyCompactList}

\chapter{Documentação da classe}
\hypertarget{unionelem_1_1dados}{\section{Referência à união elem\-:\-:dados}
\label{unionelem_1_1dados}\index{elem\-::dados@{elem\-::dados}}
}


{\ttfamily \#include $<$header.\-h$>$}

\subsection*{Atributos Públicos}
\begin{DoxyCompactItemize}
\item 
\hyperlink{header_8h_a469ff72408bc348b417cc0ee2cf860b0}{Dados\-P} $\ast$ \hyperlink{unionelem_1_1dados_aa60f6caec23042f133214228d9d48782}{dadosp}
\item 
\hyperlink{header_8h_ac31cca6d54fd1fce82eca72192564cd4}{Dados\-V\-H} $\ast$ \hyperlink{unionelem_1_1dados_aebdf1ceb7240100ec0890b018d503bd5}{dadosvh}
\item 
\hyperlink{header_8h_a341ac1667f3dc23635b071398592e724}{E\-S\-T\-R\-U\-T\-U\-R\-A\-\_\-\-B\-N} $\ast$ \hyperlink{unionelem_1_1dados_a61537c231156cd0e3cc59aa2b4875df8}{dadosest}
\end{DoxyCompactItemize}


\subsection{Documentação dos dados membro}
\hypertarget{unionelem_1_1dados_a61537c231156cd0e3cc59aa2b4875df8}{\index{elem\-::dados@{elem\-::dados}!dadosest@{dadosest}}
\index{dadosest@{dadosest}!elem::dados@{elem\-::dados}}
\subsubsection[{dadosest}]{\setlength{\rightskip}{0pt plus 5cm}{\bf E\-S\-T\-R\-U\-T\-U\-R\-A\-\_\-\-B\-N}$\ast$ elem\-::dados\-::dadosest}}\label{unionelem_1_1dados_a61537c231156cd0e3cc59aa2b4875df8}
\hypertarget{unionelem_1_1dados_aa60f6caec23042f133214228d9d48782}{\index{elem\-::dados@{elem\-::dados}!dadosp@{dadosp}}
\index{dadosp@{dadosp}!elem::dados@{elem\-::dados}}
\subsubsection[{dadosp}]{\setlength{\rightskip}{0pt plus 5cm}{\bf Dados\-P}$\ast$ elem\-::dados\-::dadosp}}\label{unionelem_1_1dados_aa60f6caec23042f133214228d9d48782}
\hypertarget{unionelem_1_1dados_aebdf1ceb7240100ec0890b018d503bd5}{\index{elem\-::dados@{elem\-::dados}!dadosvh@{dadosvh}}
\index{dadosvh@{dadosvh}!elem::dados@{elem\-::dados}}
\subsubsection[{dadosvh}]{\setlength{\rightskip}{0pt plus 5cm}{\bf Dados\-V\-H}$\ast$ elem\-::dados\-::dadosvh}}\label{unionelem_1_1dados_aebdf1ceb7240100ec0890b018d503bd5}


A documentação para esta união foi gerada a partir do seguinte ficheiro\-:\begin{DoxyCompactItemize}
\item 
\hyperlink{header_8h}{header.\-h}\end{DoxyCompactItemize}

\hypertarget{structdadosp}{\section{Referência à estrutura dadosp}
\label{structdadosp}\index{dadosp@{dadosp}}
}


{\ttfamily \#include $<$header.\-h$>$}

\subsection*{Atributos Públicos}
\begin{DoxyCompactItemize}
\item 
int \hyperlink{structdadosp_afdcec30f68f5df59dc64ad1ac705bce4}{l}
\item 
int \hyperlink{structdadosp_a22329b86bae357b8d0cd797037d91573}{c}
\item 
char \hyperlink{structdadosp_a78a74cad25b2e1c3b7bce91109b6bf29}{ch}
\end{DoxyCompactItemize}


\subsection{Documentação dos dados membro}
\hypertarget{structdadosp_a22329b86bae357b8d0cd797037d91573}{\index{dadosp@{dadosp}!c@{c}}
\index{c@{c}!dadosp@{dadosp}}
\subsubsection[{c}]{\setlength{\rightskip}{0pt plus 5cm}int dadosp\-::c}}\label{structdadosp_a22329b86bae357b8d0cd797037d91573}
\hypertarget{structdadosp_a78a74cad25b2e1c3b7bce91109b6bf29}{\index{dadosp@{dadosp}!ch@{ch}}
\index{ch@{ch}!dadosp@{dadosp}}
\subsubsection[{ch}]{\setlength{\rightskip}{0pt plus 5cm}char dadosp\-::ch}}\label{structdadosp_a78a74cad25b2e1c3b7bce91109b6bf29}
\hypertarget{structdadosp_afdcec30f68f5df59dc64ad1ac705bce4}{\index{dadosp@{dadosp}!l@{l}}
\index{l@{l}!dadosp@{dadosp}}
\subsubsection[{l}]{\setlength{\rightskip}{0pt plus 5cm}int dadosp\-::l}}\label{structdadosp_afdcec30f68f5df59dc64ad1ac705bce4}


A documentação para esta estrutura foi gerada a partir do seguinte ficheiro\-:\begin{DoxyCompactItemize}
\item 
\hyperlink{header_8h}{header.\-h}\end{DoxyCompactItemize}

\hypertarget{structdadosvh}{\section{Referência à estrutura dadosvh}
\label{structdadosvh}\index{dadosvh@{dadosvh}}
}


{\ttfamily \#include $<$header.\-h$>$}

\subsection*{Atributos Públicos}
\begin{DoxyCompactItemize}
\item 
int \hyperlink{structdadosvh_aa9f141b7d4d66748afee78fb10a8363d}{x}
\item 
int \hyperlink{structdadosvh_a5aa921bdfa9d82ac22f87c45bbbcdc30}{ant} \mbox{[}\hyperlink{header_8h_a0592dba56693fad79136250c11e5a7fe}{M\-A\-X\-\_\-\-S\-I\-Z\-E}\mbox{]}
\end{DoxyCompactItemize}


\subsection{Documentação dos dados membro}
\hypertarget{structdadosvh_a5aa921bdfa9d82ac22f87c45bbbcdc30}{\index{dadosvh@{dadosvh}!ant@{ant}}
\index{ant@{ant}!dadosvh@{dadosvh}}
\subsubsection[{ant}]{\setlength{\rightskip}{0pt plus 5cm}int dadosvh\-::ant\mbox{[}{\bf M\-A\-X\-\_\-\-S\-I\-Z\-E}\mbox{]}}}\label{structdadosvh_a5aa921bdfa9d82ac22f87c45bbbcdc30}
\hypertarget{structdadosvh_aa9f141b7d4d66748afee78fb10a8363d}{\index{dadosvh@{dadosvh}!x@{x}}
\index{x@{x}!dadosvh@{dadosvh}}
\subsubsection[{x}]{\setlength{\rightskip}{0pt plus 5cm}int dadosvh\-::x}}\label{structdadosvh_aa9f141b7d4d66748afee78fb10a8363d}


A documentação para esta estrutura foi gerada a partir do seguinte ficheiro\-:\begin{DoxyCompactItemize}
\item 
\hyperlink{header_8h}{header.\-h}\end{DoxyCompactItemize}

\hypertarget{structelem}{\section{Referência à estrutura elem}
\label{structelem}\index{elem@{elem}}
}


{\ttfamily \#include $<$header.\-h$>$}

\subsection*{Componentes}
\begin{DoxyCompactItemize}
\item 
union \hyperlink{unionelem_1_1dados}{dados}
\end{DoxyCompactItemize}
\subsection*{Atributos Públicos}
\begin{DoxyCompactItemize}
\item 
char \hyperlink{structelem_a4d60b238c774307947c9019b3c34e1f6}{com}
\item 
union \hyperlink{unionelem_1_1dados}{elem\-::dados} \hyperlink{structelem_ab01d842326a8e8bff4f514bf673173f8}{Dados}
\end{DoxyCompactItemize}


\subsection{Documentação dos dados membro}
\hypertarget{structelem_a4d60b238c774307947c9019b3c34e1f6}{\index{elem@{elem}!com@{com}}
\index{com@{com}!elem@{elem}}
\subsubsection[{com}]{\setlength{\rightskip}{0pt plus 5cm}char elem\-::com}}\label{structelem_a4d60b238c774307947c9019b3c34e1f6}
\hypertarget{structelem_ab01d842326a8e8bff4f514bf673173f8}{\index{elem@{elem}!Dados@{Dados}}
\index{Dados@{Dados}!elem@{elem}}
\subsubsection[{Dados}]{\setlength{\rightskip}{0pt plus 5cm}union {\bf elem\-::dados} elem\-::\-Dados}}\label{structelem_ab01d842326a8e8bff4f514bf673173f8}


A documentação para esta estrutura foi gerada a partir do seguinte ficheiro\-:\begin{DoxyCompactItemize}
\item 
\hyperlink{header_8h}{header.\-h}\end{DoxyCompactItemize}

\hypertarget{structest__bn}{\section{Referência à estrutura est\-\_\-bn}
\label{structest__bn}\index{est\-\_\-bn@{est\-\_\-bn}}
}


{\ttfamily \#include $<$header.\-h$>$}

\subsection*{Atributos Públicos}
\begin{DoxyCompactItemize}
\item 
int \hyperlink{structest__bn_af15be3a2b8dd97b2f754471aa655dc6f}{numlinhas}
\item 
int \hyperlink{structest__bn_a0955bffd5d0a13407e1bb2657c296a24}{numcolunas}
\item 
int \hyperlink{structest__bn_a08d8030c339bf47ead4990d55b03be3f}{segmentos\-\_\-linhas} \mbox{[}\hyperlink{header_8h_a7dd3447b4515b0ecec421b5a18ff8e97}{M\-A\-X\-\_\-\-T\-A\-B}\mbox{]}
\item 
int \hyperlink{structest__bn_a265ad1c59b326360f2f973a120f21327}{segmentos\-\_\-colunas} \mbox{[}\hyperlink{header_8h_a7dd3447b4515b0ecec421b5a18ff8e97}{M\-A\-X\-\_\-\-T\-A\-B}\mbox{]}
\item 
char \hyperlink{structest__bn_a99b2239f6adcd4c4e268639bb1041d5a}{tabuleiro} \mbox{[}\hyperlink{header_8h_a7dd3447b4515b0ecec421b5a18ff8e97}{M\-A\-X\-\_\-\-T\-A\-B}\mbox{]}\mbox{[}\hyperlink{header_8h_a7dd3447b4515b0ecec421b5a18ff8e97}{M\-A\-X\-\_\-\-T\-A\-B}\mbox{]}
\end{DoxyCompactItemize}


\subsection{Descrição detalhada}
Estrutura definida para o tabuleiro. 

\subsection{Documentação dos dados membro}
\hypertarget{structest__bn_a0955bffd5d0a13407e1bb2657c296a24}{\index{est\-\_\-bn@{est\-\_\-bn}!numcolunas@{numcolunas}}
\index{numcolunas@{numcolunas}!est_bn@{est\-\_\-bn}}
\subsubsection[{numcolunas}]{\setlength{\rightskip}{0pt plus 5cm}int est\-\_\-bn\-::numcolunas}}\label{structest__bn_a0955bffd5d0a13407e1bb2657c296a24}
\hypertarget{structest__bn_af15be3a2b8dd97b2f754471aa655dc6f}{\index{est\-\_\-bn@{est\-\_\-bn}!numlinhas@{numlinhas}}
\index{numlinhas@{numlinhas}!est_bn@{est\-\_\-bn}}
\subsubsection[{numlinhas}]{\setlength{\rightskip}{0pt plus 5cm}int est\-\_\-bn\-::numlinhas}}\label{structest__bn_af15be3a2b8dd97b2f754471aa655dc6f}
\hypertarget{structest__bn_a265ad1c59b326360f2f973a120f21327}{\index{est\-\_\-bn@{est\-\_\-bn}!segmentos\-\_\-colunas@{segmentos\-\_\-colunas}}
\index{segmentos\-\_\-colunas@{segmentos\-\_\-colunas}!est_bn@{est\-\_\-bn}}
\subsubsection[{segmentos\-\_\-colunas}]{\setlength{\rightskip}{0pt plus 5cm}int est\-\_\-bn\-::segmentos\-\_\-colunas\mbox{[}{\bf M\-A\-X\-\_\-\-T\-A\-B}\mbox{]}}}\label{structest__bn_a265ad1c59b326360f2f973a120f21327}
\hypertarget{structest__bn_a08d8030c339bf47ead4990d55b03be3f}{\index{est\-\_\-bn@{est\-\_\-bn}!segmentos\-\_\-linhas@{segmentos\-\_\-linhas}}
\index{segmentos\-\_\-linhas@{segmentos\-\_\-linhas}!est_bn@{est\-\_\-bn}}
\subsubsection[{segmentos\-\_\-linhas}]{\setlength{\rightskip}{0pt plus 5cm}int est\-\_\-bn\-::segmentos\-\_\-linhas\mbox{[}{\bf M\-A\-X\-\_\-\-T\-A\-B}\mbox{]}}}\label{structest__bn_a08d8030c339bf47ead4990d55b03be3f}
\hypertarget{structest__bn_a99b2239f6adcd4c4e268639bb1041d5a}{\index{est\-\_\-bn@{est\-\_\-bn}!tabuleiro@{tabuleiro}}
\index{tabuleiro@{tabuleiro}!est_bn@{est\-\_\-bn}}
\subsubsection[{tabuleiro}]{\setlength{\rightskip}{0pt plus 5cm}char est\-\_\-bn\-::tabuleiro\mbox{[}{\bf M\-A\-X\-\_\-\-T\-A\-B}\mbox{]}\mbox{[}{\bf M\-A\-X\-\_\-\-T\-A\-B}\mbox{]}}}\label{structest__bn_a99b2239f6adcd4c4e268639bb1041d5a}


A documentação para esta estrutura foi gerada a partir do seguinte ficheiro\-:\begin{DoxyCompactItemize}
\item 
\hyperlink{header_8h}{header.\-h}\end{DoxyCompactItemize}

\hypertarget{structstack}{\section{Referência à estrutura stack}
\label{structstack}\index{stack@{stack}}
}


{\ttfamily \#include $<$header.\-h$>$}

\subsection*{Atributos Públicos}
\begin{DoxyCompactItemize}
\item 
\hyperlink{header_8h_a04f4375bfb1c91e8c3ea010d913021ff}{E\-L\-E\-M} \hyperlink{structstack_a92a899dc86ffc25076cc16a0ed61ca8b}{estado}
\item 
struct \hyperlink{structstack}{stack} $\ast$ \hyperlink{structstack_ad00ef6fe84e682c6ac405873709da502}{prox}
\end{DoxyCompactItemize}


\subsection{Documentação dos dados membro}
\hypertarget{structstack_a92a899dc86ffc25076cc16a0ed61ca8b}{\index{stack@{stack}!estado@{estado}}
\index{estado@{estado}!stack@{stack}}
\subsubsection[{estado}]{\setlength{\rightskip}{0pt plus 5cm}{\bf E\-L\-E\-M} stack\-::estado}}\label{structstack_a92a899dc86ffc25076cc16a0ed61ca8b}
\hypertarget{structstack_ad00ef6fe84e682c6ac405873709da502}{\index{stack@{stack}!prox@{prox}}
\index{prox@{prox}!stack@{stack}}
\subsubsection[{prox}]{\setlength{\rightskip}{0pt plus 5cm}struct {\bf stack}$\ast$ stack\-::prox}}\label{structstack_ad00ef6fe84e682c6ac405873709da502}


A documentação para esta estrutura foi gerada a partir do seguinte ficheiro\-:\begin{DoxyCompactItemize}
\item 
\hyperlink{header_8h}{header.\-h}\end{DoxyCompactItemize}

\chapter{Documentação do ficheiro}
\hypertarget{header_8h}{\section{Referência ao ficheiro header.\-h}
\label{header_8h}\index{header.\-h@{header.\-h}}
}
\subsection*{Componentes}
\begin{DoxyCompactItemize}
\item 
struct \hyperlink{structest__bn}{est\-\_\-bn}
\item 
struct \hyperlink{structdadosvh}{dadosvh}
\item 
struct \hyperlink{structdadosp}{dadosp}
\item 
struct \hyperlink{structelem}{elem}
\item 
union \hyperlink{unionelem_1_1dados}{elem\-::dados}
\item 
struct \hyperlink{structstack}{stack}
\end{DoxyCompactItemize}
\subsection*{Macros}
\begin{DoxyCompactItemize}
\item 
\#define \hyperlink{header_8h_a0592dba56693fad79136250c11e5a7fe}{M\-A\-X\-\_\-\-S\-I\-Z\-E}~1024
\item 
\#define \hyperlink{header_8h_a7dd3447b4515b0ecec421b5a18ff8e97}{M\-A\-X\-\_\-\-T\-A\-B}~100
\end{DoxyCompactItemize}
\subsection*{Definições de tipos}
\begin{DoxyCompactItemize}
\item 
typedef struct \hyperlink{structest__bn}{est\-\_\-bn} \hyperlink{header_8h_a341ac1667f3dc23635b071398592e724}{E\-S\-T\-R\-U\-T\-U\-R\-A\-\_\-\-B\-N}
\item 
typedef struct \hyperlink{structdadosvh}{dadosvh} \hyperlink{header_8h_ac31cca6d54fd1fce82eca72192564cd4}{Dados\-V\-H}
\item 
typedef struct \hyperlink{structdadosp}{dadosp} \hyperlink{header_8h_a469ff72408bc348b417cc0ee2cf860b0}{Dados\-P}
\item 
typedef struct \hyperlink{structelem}{elem} \hyperlink{header_8h_a04f4375bfb1c91e8c3ea010d913021ff}{E\-L\-E\-M}
\item 
typedef struct \hyperlink{structstack}{stack} \hyperlink{header_8h_ade8bb1f37e07eb511f3873082665d739}{S\-T\-A\-C\-K}
\item 
typedef \hyperlink{header_8h_ade8bb1f37e07eb511f3873082665d739}{S\-T\-A\-C\-K} $\ast$ \hyperlink{header_8h_a6209c6f0cbf77a7146be05f6df115ceb}{stck}
\end{DoxyCompactItemize}
\subsection*{Funções}
\begin{DoxyCompactItemize}
\item 
void \hyperlink{header_8h_ab388a875e4fa77316ae653ff71a6214e}{push} (\hyperlink{header_8h_a6209c6f0cbf77a7146be05f6df115ceb}{stck} $\ast$, char, union dados)
\item 
void \hyperlink{header_8h_aabbb8b7ed90726e3bf98dd13d57b54b3}{pop} (\hyperlink{header_8h_a6209c6f0cbf77a7146be05f6df115ceb}{stck} $\ast$)
\item 
void \hyperlink{header_8h_af08d169f9b572a9ac298559fc5b15cb2}{interpretador} (\hyperlink{header_8h_a341ac1667f3dc23635b071398592e724}{E\-S\-T\-R\-U\-T\-U\-R\-A\-\_\-\-B\-N} $\ast$, \hyperlink{header_8h_a6209c6f0cbf77a7146be05f6df115ceb}{stck} $\ast$)
\item 
int \hyperlink{header_8h_a60c8bccf0a0a5948f9edf6c4ed1c145c}{interpretar} (\hyperlink{header_8h_a341ac1667f3dc23635b071398592e724}{E\-S\-T\-R\-U\-T\-U\-R\-A\-\_\-\-B\-N} $\ast$, char $\ast$, \hyperlink{header_8h_a6209c6f0cbf77a7146be05f6df115ceb}{stck} $\ast$)
\item 
int \hyperlink{header_8h_a5f064f23e852b99b119c72efcffefe05}{cmd\-\_\-c} (\hyperlink{header_8h_a341ac1667f3dc23635b071398592e724}{E\-S\-T\-R\-U\-T\-U\-R\-A\-\_\-\-B\-N} $\ast$, \hyperlink{header_8h_a6209c6f0cbf77a7146be05f6df115ceb}{stck} $\ast$)
\item 
int \hyperlink{header_8h_a63387ebbb1c8ce6d49fd64bcc9fc6714}{cmd\-\_\-m} (\hyperlink{header_8h_a341ac1667f3dc23635b071398592e724}{E\-S\-T\-R\-U\-T\-U\-R\-A\-\_\-\-B\-N} $\ast$)
\item 
int \hyperlink{header_8h_aa1527651585d5c0b5a088a16870a89b4}{cmd\-\_\-h} (\hyperlink{header_8h_a341ac1667f3dc23635b071398592e724}{E\-S\-T\-R\-U\-T\-U\-R\-A\-\_\-\-B\-N} $\ast$, int, \hyperlink{header_8h_a6209c6f0cbf77a7146be05f6df115ceb}{stck} $\ast$)
\item 
int \hyperlink{header_8h_acdbb71c71625fe05828daf7d5eecdea3}{cmd\-\_\-v} (\hyperlink{header_8h_a341ac1667f3dc23635b071398592e724}{E\-S\-T\-R\-U\-T\-U\-R\-A\-\_\-\-B\-N} $\ast$, int, \hyperlink{header_8h_a6209c6f0cbf77a7146be05f6df115ceb}{stck} $\ast$)
\item 
\hyperlink{header_8h_a469ff72408bc348b417cc0ee2cf860b0}{Dados\-P} $\ast$ \hyperlink{header_8h_a50a30835711378fc3a6669f18f18c1d4}{new\-Dados\-P} (int, int, char)
\item 
int \hyperlink{header_8h_a1692b088daaad679b0cd8809303646c0}{cmd\-\_\-p} (\hyperlink{header_8h_a341ac1667f3dc23635b071398592e724}{E\-S\-T\-R\-U\-T\-U\-R\-A\-\_\-\-B\-N} $\ast$, char, int, int, \hyperlink{header_8h_a6209c6f0cbf77a7146be05f6df115ceb}{stck} $\ast$)
\item 
int \hyperlink{header_8h_aba46e466d2429fabadb26cf17ff2c6df}{cmd\-\_\-l} (\hyperlink{header_8h_a341ac1667f3dc23635b071398592e724}{E\-S\-T\-R\-U\-T\-U\-R\-A\-\_\-\-B\-N} $\ast$, char $\ast$, \hyperlink{header_8h_a6209c6f0cbf77a7146be05f6df115ceb}{stck} $\ast$)
\item 
int \hyperlink{header_8h_afe1808fe4cd6603c7cdd318e4192289e}{cmd\-\_\-e} (\hyperlink{header_8h_a341ac1667f3dc23635b071398592e724}{E\-S\-T\-R\-U\-T\-U\-R\-A\-\_\-\-B\-N} $\ast$, char $\ast$)
\item 
int \hyperlink{header_8h_afe6790c34e10764ba4ceaed8103a4e9d}{cmd\-\_\-\-V} (\hyperlink{header_8h_a341ac1667f3dc23635b071398592e724}{E\-S\-T\-R\-U\-T\-U\-R\-A\-\_\-\-B\-N} $\ast$)
\item 
int \hyperlink{header_8h_a5fb6f50cbc25a556b19bc095533f6393}{verifica\-Agua} (\hyperlink{header_8h_a341ac1667f3dc23635b071398592e724}{E\-S\-T\-R\-U\-T\-U\-R\-A\-\_\-\-B\-N} $\ast$)
\item 
int \hyperlink{header_8h_a172638f19aff4f9bd3a018a16b5f98e8}{verifica\-Barcos} (\hyperlink{header_8h_a341ac1667f3dc23635b071398592e724}{E\-S\-T\-R\-U\-T\-U\-R\-A\-\_\-\-B\-N} $\ast$)
\item 
int \hyperlink{header_8h_a3f4806bc2a02a5a364967823141b10ac}{verifica\-Cantos\-P} (\hyperlink{header_8h_a341ac1667f3dc23635b071398592e724}{E\-S\-T\-R\-U\-T\-U\-R\-A\-\_\-\-B\-N} $\ast$)
\item 
int \hyperlink{header_8h_a8ab3436eacf68a0f7a7a31f2c7d13397}{verifica\-S\-U\-B} (\hyperlink{header_8h_a341ac1667f3dc23635b071398592e724}{E\-S\-T\-R\-U\-T\-U\-R\-A\-\_\-\-B\-N} $\ast$, char, int, int)
\item 
int \hyperlink{header_8h_a84da85ba6c2068180f8c976bd4ccbea5}{verifica\-Seg} (\hyperlink{header_8h_a341ac1667f3dc23635b071398592e724}{E\-S\-T\-R\-U\-T\-U\-R\-A\-\_\-\-B\-N} $\ast$)
\item 
int \hyperlink{header_8h_ac3996c0889a577d3df4d672fbd4d5b95}{verifica\-Peca\-Ex} (\hyperlink{header_8h_a341ac1667f3dc23635b071398592e724}{E\-S\-T\-R\-U\-T\-U\-R\-A\-\_\-\-B\-N} $\ast$)
\item 
int \hyperlink{header_8h_ab79e61e27854e265b504ade42f5d6ad8}{cmd\-\_\-\-E1} (\hyperlink{header_8h_a341ac1667f3dc23635b071398592e724}{E\-S\-T\-R\-U\-T\-U\-R\-A\-\_\-\-B\-N} $\ast$, \hyperlink{header_8h_a6209c6f0cbf77a7146be05f6df115ceb}{stck} $\ast$)
\item 
int \hyperlink{header_8h_ae2c68ce3d86a30dde2be99697b1d2533}{cmd\-\_\-\-E2} (\hyperlink{header_8h_a341ac1667f3dc23635b071398592e724}{E\-S\-T\-R\-U\-T\-U\-R\-A\-\_\-\-B\-N} $\ast$, \hyperlink{header_8h_a6209c6f0cbf77a7146be05f6df115ceb}{stck} $\ast$)
\item 
int \hyperlink{header_8h_a74de530d637c553d7713b08eb2ca7b74}{E2\-L} (\hyperlink{header_8h_a341ac1667f3dc23635b071398592e724}{E\-S\-T\-R\-U\-T\-U\-R\-A\-\_\-\-B\-N} $\ast$)
\item 
void \hyperlink{header_8h_aad6d3a3fe8e10625615605555c675715}{haux} (\hyperlink{header_8h_a341ac1667f3dc23635b071398592e724}{E\-S\-T\-R\-U\-T\-U\-R\-A\-\_\-\-B\-N} $\ast$, int)
\item 
int \hyperlink{header_8h_a7c51831c11c69adde64f64bb7fd467dd}{E2\-C} (\hyperlink{header_8h_a341ac1667f3dc23635b071398592e724}{E\-S\-T\-R\-U\-T\-U\-R\-A\-\_\-\-B\-N} $\ast$)
\item 
void \hyperlink{header_8h_ad51e83dedad9fe2e52a2842db30b3434}{vaux} (\hyperlink{header_8h_a341ac1667f3dc23635b071398592e724}{E\-S\-T\-R\-U\-T\-U\-R\-A\-\_\-\-B\-N} $\ast$, int)
\item 
int \hyperlink{header_8h_aba53266076b1ba73d7dc0db01a58c9ee}{pertence} (char)
\item 
int \hyperlink{header_8h_a6ef859f2d4f5d230b6a3617936a6d9a1}{cmd\-\_\-\-E3} (\hyperlink{header_8h_a341ac1667f3dc23635b071398592e724}{E\-S\-T\-R\-U\-T\-U\-R\-A\-\_\-\-B\-N} $\ast$, \hyperlink{header_8h_a6209c6f0cbf77a7146be05f6df115ceb}{stck} $\ast$)
\item 
void \hyperlink{header_8h_a0858462671024466f8669aed58a00d81}{tiraros} (\hyperlink{header_8h_a341ac1667f3dc23635b071398592e724}{E\-S\-T\-R\-U\-T\-U\-R\-A\-\_\-\-B\-N} $\ast$)
\item 
int \hyperlink{header_8h_aae86d3206ebffbec52cd0050e79d96ee}{porsegs} (\hyperlink{header_8h_a341ac1667f3dc23635b071398592e724}{E\-S\-T\-R\-U\-T\-U\-R\-A\-\_\-\-B\-N} $\ast$)
\item 
int \hyperlink{header_8h_a7163da75079918d15bb375d69af2dfd9}{porbarcos} (\hyperlink{header_8h_a341ac1667f3dc23635b071398592e724}{E\-S\-T\-R\-U\-T\-U\-R\-A\-\_\-\-B\-N} $\ast$)
\item 
void \hyperlink{header_8h_a4c3cc8ed850019e6ec428043c717a89f}{poebarcos\-Auxc} (\hyperlink{header_8h_a341ac1667f3dc23635b071398592e724}{E\-S\-T\-R\-U\-T\-U\-R\-A\-\_\-\-B\-N} $\ast$, int, int)
\item 
void \hyperlink{header_8h_aef04d1ba329640ac64a331231e43bf6d}{caso\-Ec} (\hyperlink{header_8h_a341ac1667f3dc23635b071398592e724}{E\-S\-T\-R\-U\-T\-U\-R\-A\-\_\-\-B\-N} $\ast$, int, int)
\item 
void \hyperlink{header_8h_ad6dd9bca259525b1a28c7623aaea301c}{caso\-El} (\hyperlink{header_8h_a341ac1667f3dc23635b071398592e724}{E\-S\-T\-R\-U\-T\-U\-R\-A\-\_\-\-B\-N} $\ast$, int, int)
\item 
void \hyperlink{header_8h_a6e802f4810aae2a030eb414236840023}{poebarcos\-Auxl} (\hyperlink{header_8h_a341ac1667f3dc23635b071398592e724}{E\-S\-T\-R\-U\-T\-U\-R\-A\-\_\-\-B\-N} $\ast$, int, int)
\item 
int \hyperlink{header_8h_aa7af588d3c0710101b7fd4563e564982}{porsubs} (\hyperlink{header_8h_a341ac1667f3dc23635b071398592e724}{E\-S\-T\-R\-U\-T\-U\-R\-A\-\_\-\-B\-N} $\ast$)
\item 
void \hyperlink{header_8h_a3751d683bcb249d0b7d1dcf65b6c5d97}{poros} (\hyperlink{header_8h_a341ac1667f3dc23635b071398592e724}{E\-S\-T\-R\-U\-T\-U\-R\-A\-\_\-\-B\-N} $\ast$)
\item 
void \hyperlink{header_8h_a7990488aaeb3798cea2a7cb34f5f3945}{porosc} (\hyperlink{header_8h_a341ac1667f3dc23635b071398592e724}{E\-S\-T\-R\-U\-T\-U\-R\-A\-\_\-\-B\-N} $\ast$)
\item 
int \hyperlink{header_8h_ad1017c7d86f4155e2b7a54370a6d2505}{contapc} (\hyperlink{header_8h_a341ac1667f3dc23635b071398592e724}{E\-S\-T\-R\-U\-T\-U\-R\-A\-\_\-\-B\-N} $\ast$, int)
\item 
int \hyperlink{header_8h_ae5c776561dc50e0c46b28563ef8bba5d}{contasegc} (\hyperlink{header_8h_a341ac1667f3dc23635b071398592e724}{E\-S\-T\-R\-U\-T\-U\-R\-A\-\_\-\-B\-N} $\ast$, int)
\item 
void \hyperlink{header_8h_a0d3e599ef4a848732e19b97da6696ae7}{porosl} (\hyperlink{header_8h_a341ac1667f3dc23635b071398592e724}{E\-S\-T\-R\-U\-T\-U\-R\-A\-\_\-\-B\-N} $\ast$)
\item 
int \hyperlink{header_8h_a4b82895cd2d62612c5f4d6b70cc38859}{contapl} (\hyperlink{header_8h_a341ac1667f3dc23635b071398592e724}{E\-S\-T\-R\-U\-T\-U\-R\-A\-\_\-\-B\-N} $\ast$, int)
\item 
int \hyperlink{header_8h_a75ba6d65db66767f9bdafe6ed1f59146}{contasegl} (\hyperlink{header_8h_a341ac1667f3dc23635b071398592e724}{E\-S\-T\-R\-U\-T\-U\-R\-A\-\_\-\-B\-N} $\ast$, int)
\item 
int \hyperlink{header_8h_a6e5986b8a99d54b0edd7f5d292cf9e7c}{pertence3} (char)
\item 
void \hyperlink{header_8h_a3891e7cfa8326204b890100cc4415901}{E1\-\_\-filtraresto} (\hyperlink{header_8h_a341ac1667f3dc23635b071398592e724}{E\-S\-T\-R\-U\-T\-U\-R\-A\-\_\-\-B\-N} $\ast$)
\item 
void \hyperlink{header_8h_aa2e5bd15a8e36136d2e93f7152d18b31}{E1\-\_\-primlinha} (\hyperlink{header_8h_a341ac1667f3dc23635b071398592e724}{E\-S\-T\-R\-U\-T\-U\-R\-A\-\_\-\-B\-N} $\ast$)
\item 
int \hyperlink{header_8h_a9e6501a1db374bad0edc521175045db2}{contaaguas} (\hyperlink{header_8h_a341ac1667f3dc23635b071398592e724}{E\-S\-T\-R\-U\-T\-U\-R\-A\-\_\-\-B\-N} $\ast$)
\item 
void \hyperlink{header_8h_adeb57da836f612ad2be044bc164230b0}{cmd\-\_\-\-D} (\hyperlink{header_8h_a341ac1667f3dc23635b071398592e724}{E\-S\-T\-R\-U\-T\-U\-R\-A\-\_\-\-B\-N} $\ast$, \hyperlink{header_8h_a6209c6f0cbf77a7146be05f6df115ceb}{stck} $\ast$)
\item 
void \hyperlink{header_8h_aebfff5a330995069c917ea559880f1cc}{undo\-\_\-p} (\hyperlink{header_8h_a341ac1667f3dc23635b071398592e724}{E\-S\-T\-R\-U\-T\-U\-R\-A\-\_\-\-B\-N} $\ast$, \hyperlink{header_8h_a6209c6f0cbf77a7146be05f6df115ceb}{stck} $\ast$)
\item 
void \hyperlink{header_8h_a3eb15c46ca4a0bbf5f8583106e31443a}{undo\-\_\-v} (\hyperlink{header_8h_a341ac1667f3dc23635b071398592e724}{E\-S\-T\-R\-U\-T\-U\-R\-A\-\_\-\-B\-N} $\ast$, \hyperlink{header_8h_a6209c6f0cbf77a7146be05f6df115ceb}{stck} $\ast$)
\item 
void \hyperlink{header_8h_ae62b88899707025f3893fd2939b4e3de}{undo\-\_\-h} (\hyperlink{header_8h_a341ac1667f3dc23635b071398592e724}{E\-S\-T\-R\-U\-T\-U\-R\-A\-\_\-\-B\-N} $\ast$, \hyperlink{header_8h_a6209c6f0cbf77a7146be05f6df115ceb}{stck} $\ast$)
\item 
void \hyperlink{header_8h_a512a6ef1c8fef57ded24e2f7adb5347c}{undo\-\_\-\-C\-L\-E} (\hyperlink{header_8h_a341ac1667f3dc23635b071398592e724}{E\-S\-T\-R\-U\-T\-U\-R\-A\-\_\-\-B\-N} $\ast$, \hyperlink{header_8h_a6209c6f0cbf77a7146be05f6df115ceb}{stck} $\ast$)
\item 
void \hyperlink{header_8h_a7fb1b6d7086e9d83254f415f6202ac2f}{cmd\-\_\-\-R} (\hyperlink{header_8h_a341ac1667f3dc23635b071398592e724}{E\-S\-T\-R\-U\-T\-U\-R\-A\-\_\-\-B\-N} $\ast$, \hyperlink{header_8h_a6209c6f0cbf77a7146be05f6df115ceb}{stck} $\ast$)
\item 
\hyperlink{header_8h_ac31cca6d54fd1fce82eca72192564cd4}{Dados\-V\-H} $\ast$ \hyperlink{header_8h_aa5babe6829adc82772460605370afade}{new\-Dados\-V\-H} (int, int, int\mbox{[}$\,$\mbox{]})
\item 
\hyperlink{header_8h_a341ac1667f3dc23635b071398592e724}{E\-S\-T\-R\-U\-T\-U\-R\-A\-\_\-\-B\-N} $\ast$ \hyperlink{header_8h_afed3fbcd069873008da84889cc81a19a}{new\-Dados\-Est} (\hyperlink{header_8h_a341ac1667f3dc23635b071398592e724}{E\-S\-T\-R\-U\-T\-U\-R\-A\-\_\-\-B\-N} $\ast$)
\end{DoxyCompactItemize}


\subsection{Documentação das macros}
\hypertarget{header_8h_a0592dba56693fad79136250c11e5a7fe}{\index{header.\-h@{header.\-h}!M\-A\-X\-\_\-\-S\-I\-Z\-E@{M\-A\-X\-\_\-\-S\-I\-Z\-E}}
\index{M\-A\-X\-\_\-\-S\-I\-Z\-E@{M\-A\-X\-\_\-\-S\-I\-Z\-E}!header.h@{header.\-h}}
\subsubsection[{M\-A\-X\-\_\-\-S\-I\-Z\-E}]{\setlength{\rightskip}{0pt plus 5cm}\#define M\-A\-X\-\_\-\-S\-I\-Z\-E~1024}}\label{header_8h_a0592dba56693fad79136250c11e5a7fe}
\hypertarget{header_8h_a7dd3447b4515b0ecec421b5a18ff8e97}{\index{header.\-h@{header.\-h}!M\-A\-X\-\_\-\-T\-A\-B@{M\-A\-X\-\_\-\-T\-A\-B}}
\index{M\-A\-X\-\_\-\-T\-A\-B@{M\-A\-X\-\_\-\-T\-A\-B}!header.h@{header.\-h}}
\subsubsection[{M\-A\-X\-\_\-\-T\-A\-B}]{\setlength{\rightskip}{0pt plus 5cm}\#define M\-A\-X\-\_\-\-T\-A\-B~100}}\label{header_8h_a7dd3447b4515b0ecec421b5a18ff8e97}


\subsection{Documentação dos tipos}
\hypertarget{header_8h_a469ff72408bc348b417cc0ee2cf860b0}{\index{header.\-h@{header.\-h}!Dados\-P@{Dados\-P}}
\index{Dados\-P@{Dados\-P}!header.h@{header.\-h}}
\subsubsection[{Dados\-P}]{\setlength{\rightskip}{0pt plus 5cm}typedef struct {\bf dadosp} {\bf Dados\-P}}}\label{header_8h_a469ff72408bc348b417cc0ee2cf860b0}
\hypertarget{header_8h_ac31cca6d54fd1fce82eca72192564cd4}{\index{header.\-h@{header.\-h}!Dados\-V\-H@{Dados\-V\-H}}
\index{Dados\-V\-H@{Dados\-V\-H}!header.h@{header.\-h}}
\subsubsection[{Dados\-V\-H}]{\setlength{\rightskip}{0pt plus 5cm}typedef struct {\bf dadosvh} {\bf Dados\-V\-H}}}\label{header_8h_ac31cca6d54fd1fce82eca72192564cd4}
\hypertarget{header_8h_a04f4375bfb1c91e8c3ea010d913021ff}{\index{header.\-h@{header.\-h}!E\-L\-E\-M@{E\-L\-E\-M}}
\index{E\-L\-E\-M@{E\-L\-E\-M}!header.h@{header.\-h}}
\subsubsection[{E\-L\-E\-M}]{\setlength{\rightskip}{0pt plus 5cm}typedef struct {\bf elem} {\bf E\-L\-E\-M}}}\label{header_8h_a04f4375bfb1c91e8c3ea010d913021ff}
\hypertarget{header_8h_a341ac1667f3dc23635b071398592e724}{\index{header.\-h@{header.\-h}!E\-S\-T\-R\-U\-T\-U\-R\-A\-\_\-\-B\-N@{E\-S\-T\-R\-U\-T\-U\-R\-A\-\_\-\-B\-N}}
\index{E\-S\-T\-R\-U\-T\-U\-R\-A\-\_\-\-B\-N@{E\-S\-T\-R\-U\-T\-U\-R\-A\-\_\-\-B\-N}!header.h@{header.\-h}}
\subsubsection[{E\-S\-T\-R\-U\-T\-U\-R\-A\-\_\-\-B\-N}]{\setlength{\rightskip}{0pt plus 5cm}typedef struct {\bf est\-\_\-bn} {\bf E\-S\-T\-R\-U\-T\-U\-R\-A\-\_\-\-B\-N}}}\label{header_8h_a341ac1667f3dc23635b071398592e724}
Estrutura definida para o tabuleiro. \hypertarget{header_8h_ade8bb1f37e07eb511f3873082665d739}{\index{header.\-h@{header.\-h}!S\-T\-A\-C\-K@{S\-T\-A\-C\-K}}
\index{S\-T\-A\-C\-K@{S\-T\-A\-C\-K}!header.h@{header.\-h}}
\subsubsection[{S\-T\-A\-C\-K}]{\setlength{\rightskip}{0pt plus 5cm}typedef struct {\bf stack} {\bf S\-T\-A\-C\-K}}}\label{header_8h_ade8bb1f37e07eb511f3873082665d739}
\hypertarget{header_8h_a6209c6f0cbf77a7146be05f6df115ceb}{\index{header.\-h@{header.\-h}!stck@{stck}}
\index{stck@{stck}!header.h@{header.\-h}}
\subsubsection[{stck}]{\setlength{\rightskip}{0pt plus 5cm}typedef {\bf S\-T\-A\-C\-K}$\ast$ {\bf stck}}}\label{header_8h_a6209c6f0cbf77a7146be05f6df115ceb}


\subsection{Documentação das funções}
\hypertarget{header_8h_aef04d1ba329640ac64a331231e43bf6d}{\index{header.\-h@{header.\-h}!caso\-Ec@{caso\-Ec}}
\index{caso\-Ec@{caso\-Ec}!header.h@{header.\-h}}
\subsubsection[{caso\-Ec}]{\setlength{\rightskip}{0pt plus 5cm}void caso\-Ec (
\begin{DoxyParamCaption}
\item[{{\bf E\-S\-T\-R\-U\-T\-U\-R\-A\-\_\-\-B\-N} $\ast$}]{, }
\item[{int}]{, }
\item[{int}]{}
\end{DoxyParamCaption}
)}}\label{header_8h_aef04d1ba329640ac64a331231e43bf6d}
\hypertarget{header_8h_ad6dd9bca259525b1a28c7623aaea301c}{\index{header.\-h@{header.\-h}!caso\-El@{caso\-El}}
\index{caso\-El@{caso\-El}!header.h@{header.\-h}}
\subsubsection[{caso\-El}]{\setlength{\rightskip}{0pt plus 5cm}void caso\-El (
\begin{DoxyParamCaption}
\item[{{\bf E\-S\-T\-R\-U\-T\-U\-R\-A\-\_\-\-B\-N} $\ast$}]{, }
\item[{int}]{, }
\item[{int}]{}
\end{DoxyParamCaption}
)}}\label{header_8h_ad6dd9bca259525b1a28c7623aaea301c}
\hypertarget{header_8h_a5f064f23e852b99b119c72efcffefe05}{\index{header.\-h@{header.\-h}!cmd\-\_\-c@{cmd\-\_\-c}}
\index{cmd\-\_\-c@{cmd\-\_\-c}!header.h@{header.\-h}}
\subsubsection[{cmd\-\_\-c}]{\setlength{\rightskip}{0pt plus 5cm}int cmd\-\_\-c (
\begin{DoxyParamCaption}
\item[{{\bf E\-S\-T\-R\-U\-T\-U\-R\-A\-\_\-\-B\-N} $\ast$}]{est\-\_\-bn, }
\item[{{\bf stck} $\ast$}]{stack}
\end{DoxyParamCaption}
)}}\label{header_8h_a5f064f23e852b99b119c72efcffefe05}
O comando c lê o tabuleiro a partir do standard input.


\begin{DoxyParams}{Parâmetros}
{\em A} & função vai receber um tabuleiro a partir do standard input. \\
\hline
{\em Recebe} & a estrutura (\hyperlink{structest__bn}{est\-\_\-bn}), dois arrays um linha (recebe a linha do tabuleiro) e restolinha (guarda o resto da linha). \\
\hline
{\em lin} & e col são as duas variáveis para o tamanho das linhas e colunas respetivamente.\\
\hline
\end{DoxyParams}
\begin{DoxyReturn}{Retorna}
Não retorna nada apenas lê o tabuleiro a partir do teclado. 
\end{DoxyReturn}
\hypertarget{header_8h_adeb57da836f612ad2be044bc164230b0}{\index{header.\-h@{header.\-h}!cmd\-\_\-\-D@{cmd\-\_\-\-D}}
\index{cmd\-\_\-\-D@{cmd\-\_\-\-D}!header.h@{header.\-h}}
\subsubsection[{cmd\-\_\-\-D}]{\setlength{\rightskip}{0pt plus 5cm}void cmd\-\_\-\-D (
\begin{DoxyParamCaption}
\item[{{\bf E\-S\-T\-R\-U\-T\-U\-R\-A\-\_\-\-B\-N} $\ast$}]{est\-\_\-bn, }
\item[{{\bf stck} $\ast$}]{stack}
\end{DoxyParamCaption}
)}}\label{header_8h_adeb57da836f612ad2be044bc164230b0}
O comando D desfaz um comando e volta ao estado anterior.


\begin{DoxyParams}{Parâmetros}
{\em A} & função desfaz a última alteração feita ao tabuleiro. \\
\hline
{\em Esta} & é constituida por várias funções auxiliares pois cada comando é diferente na sua forma de alterar o tabuleiro.\\
\hline
\end{DoxyParams}
\begin{DoxyReturn}{Retorna}
Não retorna nada apenas desfaz a última alteração feita ao tabuleiro. 
\end{DoxyReturn}
\hypertarget{header_8h_afe1808fe4cd6603c7cdd318e4192289e}{\index{header.\-h@{header.\-h}!cmd\-\_\-e@{cmd\-\_\-e}}
\index{cmd\-\_\-e@{cmd\-\_\-e}!header.h@{header.\-h}}
\subsubsection[{cmd\-\_\-e}]{\setlength{\rightskip}{0pt plus 5cm}int cmd\-\_\-e (
\begin{DoxyParamCaption}
\item[{{\bf E\-S\-T\-R\-U\-T\-U\-R\-A\-\_\-\-B\-N} $\ast$}]{est\-\_\-bn, }
\item[{char $\ast$}]{path}
\end{DoxyParamCaption}
)}}\label{header_8h_afe1808fe4cd6603c7cdd318e4192289e}
O comando e escrever um tabuleiro num ficheiro .txt.


\begin{DoxyParams}{Parâmetros}
{\em A} & função vai receber um tabuleiro a partir do qual vai escrever num dado ficheiro, para o qual teremos de escolher o nome e o path.\\
\hline
\end{DoxyParams}
\begin{DoxyReturn}{Retorna}
Não retorna nada apenas escreve no ficheiro .txt. 
\end{DoxyReturn}
\hypertarget{header_8h_ab79e61e27854e265b504ade42f5d6ad8}{\index{header.\-h@{header.\-h}!cmd\-\_\-\-E1@{cmd\-\_\-\-E1}}
\index{cmd\-\_\-\-E1@{cmd\-\_\-\-E1}!header.h@{header.\-h}}
\subsubsection[{cmd\-\_\-\-E1}]{\setlength{\rightskip}{0pt plus 5cm}int cmd\-\_\-\-E1 (
\begin{DoxyParamCaption}
\item[{{\bf E\-S\-T\-R\-U\-T\-U\-R\-A\-\_\-\-B\-N} $\ast$}]{est\-\_\-bn, }
\item[{{\bf stck} $\ast$}]{stack}
\end{DoxyParamCaption}
)}}\label{header_8h_ab79e61e27854e265b504ade42f5d6ad8}
O comando E1 é a estratégia 1.


\begin{DoxyParams}{Parâmetros}
{\em A} & função recebe o tabuleiro e é responsável por colocar água onde se deduz que vai existir à volta de todos os segmentos de barcos já colocados. \\
\hline
{\em É} & constituída por duas funções auxiliares chamadas E1\-\_\-primlinha e E1\-\_\-filtraresto que começam por preencher a primeira linha. Caso existam mais do que uma linha, esta preenchê-\/las-\/á.\\
\hline
\end{DoxyParams}
\begin{DoxyReturn}{Retorna}
Retorna 1 se houverem alterações e 0 se não houverem. 
\end{DoxyReturn}
\hypertarget{header_8h_ae2c68ce3d86a30dde2be99697b1d2533}{\index{header.\-h@{header.\-h}!cmd\-\_\-\-E2@{cmd\-\_\-\-E2}}
\index{cmd\-\_\-\-E2@{cmd\-\_\-\-E2}!header.h@{header.\-h}}
\subsubsection[{cmd\-\_\-\-E2}]{\setlength{\rightskip}{0pt plus 5cm}int cmd\-\_\-\-E2 (
\begin{DoxyParamCaption}
\item[{{\bf E\-S\-T\-R\-U\-T\-U\-R\-A\-\_\-\-B\-N} $\ast$}]{est\-\_\-bn, }
\item[{{\bf stck} $\ast$}]{stack}
\end{DoxyParamCaption}
)}}\label{header_8h_ae2c68ce3d86a30dde2be99697b1d2533}
O comando E2 é a estratégia 2.


\begin{DoxyParams}{Parâmetros}
{\em A} & função recebe o tabuleiro e é responsável por colocar água nas linhas e colunas em todos os segmentos de barcos que já foram colocados.\\
\hline
\end{DoxyParams}
\begin{DoxyReturn}{Retorna}
retorna 1 se houverem alterações e 0 se não houverem. 
\end{DoxyReturn}
\hypertarget{header_8h_a6ef859f2d4f5d230b6a3617936a6d9a1}{\index{header.\-h@{header.\-h}!cmd\-\_\-\-E3@{cmd\-\_\-\-E3}}
\index{cmd\-\_\-\-E3@{cmd\-\_\-\-E3}!header.h@{header.\-h}}
\subsubsection[{cmd\-\_\-\-E3}]{\setlength{\rightskip}{0pt plus 5cm}int cmd\-\_\-\-E3 (
\begin{DoxyParamCaption}
\item[{{\bf E\-S\-T\-R\-U\-T\-U\-R\-A\-\_\-\-B\-N} $\ast$}]{est\-\_\-bn, }
\item[{{\bf stck} $\ast$}]{stack}
\end{DoxyParamCaption}
)}}\label{header_8h_a6ef859f2d4f5d230b6a3617936a6d9a1}
O comando E3 é a estratégia E3


\begin{DoxyParams}{Parâmetros}
{\em A} & função recebe um tabuleiro quase completo, ou seja onde a água já esteja completamente preenchida, e coloca segmentos de barcos nas linhas e colunas nas quais todos os espaços vazios têm que conter segmentos de barcos para que o número correspondente seja respeitado.\\
\hline
\end{DoxyParams}
\begin{DoxyReturn}{Retorna}
A funcao retorna 1 se houverem alterações e 0 se não houverem. 
\end{DoxyReturn}
\hypertarget{header_8h_aa1527651585d5c0b5a088a16870a89b4}{\index{header.\-h@{header.\-h}!cmd\-\_\-h@{cmd\-\_\-h}}
\index{cmd\-\_\-h@{cmd\-\_\-h}!header.h@{header.\-h}}
\subsubsection[{cmd\-\_\-h}]{\setlength{\rightskip}{0pt plus 5cm}int cmd\-\_\-h (
\begin{DoxyParamCaption}
\item[{{\bf E\-S\-T\-R\-U\-T\-U\-R\-A\-\_\-\-B\-N} $\ast$}]{est\-\_\-bn, }
\item[{int}]{l, }
\item[{{\bf stck} $\ast$}]{stack}
\end{DoxyParamCaption}
)}}\label{header_8h_aa1527651585d5c0b5a088a16870a89b4}
O comando h coloca um estado em todas as grelhas de uma linha.


\begin{DoxyParams}{Parâmetros}
{\em A} & função coloca o estado de todas as grelhas da linha n o num que ainda não estão determinadas como sendo água. \\
\hline
{\em Onde} & está '.' vai colocar '$\sim$' na linha correspondente.\\
\hline
\end{DoxyParams}
\begin{DoxyReturn}{Retorna}
Não retorna nada apenas altera o tabuleiro. 
\end{DoxyReturn}
\hypertarget{header_8h_aba46e466d2429fabadb26cf17ff2c6df}{\index{header.\-h@{header.\-h}!cmd\-\_\-l@{cmd\-\_\-l}}
\index{cmd\-\_\-l@{cmd\-\_\-l}!header.h@{header.\-h}}
\subsubsection[{cmd\-\_\-l}]{\setlength{\rightskip}{0pt plus 5cm}int cmd\-\_\-l (
\begin{DoxyParamCaption}
\item[{{\bf E\-S\-T\-R\-U\-T\-U\-R\-A\-\_\-\-B\-N} $\ast$}]{est\-\_\-bn, }
\item[{char $\ast$}]{path, }
\item[{{\bf stck} $\ast$}]{stack}
\end{DoxyParamCaption}
)}}\label{header_8h_aba46e466d2429fabadb26cf17ff2c6df}
O comando l lê o tabuleiro a partir de um ficheiro .txt.


\begin{DoxyParams}{Parâmetros}
{\em A} & função recebe um tabuleiro a partir do ficheiro .txt escrito que pertence à mesma diretoria onde o projeto se encontra.\\
\hline
\end{DoxyParams}
\begin{DoxyReturn}{Retorna}
Não retorna nada apenas lê o tabuleiro. 
\end{DoxyReturn}
\hypertarget{header_8h_a63387ebbb1c8ce6d49fd64bcc9fc6714}{\index{header.\-h@{header.\-h}!cmd\-\_\-m@{cmd\-\_\-m}}
\index{cmd\-\_\-m@{cmd\-\_\-m}!header.h@{header.\-h}}
\subsubsection[{cmd\-\_\-m}]{\setlength{\rightskip}{0pt plus 5cm}int cmd\-\_\-m (
\begin{DoxyParamCaption}
\item[{{\bf E\-S\-T\-R\-U\-T\-U\-R\-A\-\_\-\-B\-N} $\ast$}]{est\-\_\-bn}
\end{DoxyParamCaption}
)}}\label{header_8h_a63387ebbb1c8ce6d49fd64bcc9fc6714}
O comando m mostra o tabuleiro.


\begin{DoxyParams}{Parâmetros}
{\em A} & função vai receber um tabuleiro e vai imprimi-\/lo no ecrã. \\
\hline
{\em lin} & e col são as duas variáveis para o tamanho das linhas e colunas respetivamente.\\
\hline
\end{DoxyParams}
\begin{DoxyReturn}{Retorna}
Não retorna nada apenas mostra tabuleiro. 
\end{DoxyReturn}
\hypertarget{header_8h_a1692b088daaad679b0cd8809303646c0}{\index{header.\-h@{header.\-h}!cmd\-\_\-p@{cmd\-\_\-p}}
\index{cmd\-\_\-p@{cmd\-\_\-p}!header.h@{header.\-h}}
\subsubsection[{cmd\-\_\-p}]{\setlength{\rightskip}{0pt plus 5cm}int cmd\-\_\-p (
\begin{DoxyParamCaption}
\item[{{\bf E\-S\-T\-R\-U\-T\-U\-R\-A\-\_\-\-B\-N} $\ast$}]{est\-\_\-bn, }
\item[{char}]{x, }
\item[{int}]{l, }
\item[{int}]{c, }
\item[{{\bf stck} $\ast$}]{stack}
\end{DoxyParamCaption}
)}}\label{header_8h_a1692b088daaad679b0cd8809303646c0}
O comando p coloca uma peça de um barco ou submarino na linha e coluna dados.


\begin{DoxyParams}{Parâmetros}
{\em A} & função ao receber um carater deve colocar na linha e coluna correspondentes esse mesmo carater. \\
\hline
{\em ln} & e cl são as variáveis que correspondem à linha ln e coluna cl.\\
\hline
\end{DoxyParams}
\begin{DoxyReturn}{Retorna}
Não retorna nada apenas altera o tabuleiro. 
\end{DoxyReturn}
\hypertarget{header_8h_a7fb1b6d7086e9d83254f415f6202ac2f}{\index{header.\-h@{header.\-h}!cmd\-\_\-\-R@{cmd\-\_\-\-R}}
\index{cmd\-\_\-\-R@{cmd\-\_\-\-R}!header.h@{header.\-h}}
\subsubsection[{cmd\-\_\-\-R}]{\setlength{\rightskip}{0pt plus 5cm}void cmd\-\_\-\-R (
\begin{DoxyParamCaption}
\item[{{\bf E\-S\-T\-R\-U\-T\-U\-R\-A\-\_\-\-B\-N} $\ast$}]{est\-\_\-bn, }
\item[{{\bf stck} $\ast$}]{stack}
\end{DoxyParamCaption}
)}}\label{header_8h_a7fb1b6d7086e9d83254f415f6202ac2f}
Esta função foi criada com o intuito de resolver tabuleiros, apesar de a nossa nao estar complexa o suficiente para tal.


\begin{DoxyParams}{Parâmetros}
{\em a} & função recebe a estrutura, do tipo E\-S\-T\-R\-U\-T\-U\-R\-A\-\_\-\-B\-N. \\
\hline
{\em recebe} & também a stack com o intuito de ser possivel desfazer as alterações. \\
\hline
\end{DoxyParams}
\hypertarget{header_8h_acdbb71c71625fe05828daf7d5eecdea3}{\index{header.\-h@{header.\-h}!cmd\-\_\-v@{cmd\-\_\-v}}
\index{cmd\-\_\-v@{cmd\-\_\-v}!header.h@{header.\-h}}
\subsubsection[{cmd\-\_\-v}]{\setlength{\rightskip}{0pt plus 5cm}int cmd\-\_\-v (
\begin{DoxyParamCaption}
\item[{{\bf E\-S\-T\-R\-U\-T\-U\-R\-A\-\_\-\-B\-N} $\ast$}]{est\-\_\-bn, }
\item[{int}]{c, }
\item[{{\bf stck} $\ast$}]{stack}
\end{DoxyParamCaption}
)}}\label{header_8h_acdbb71c71625fe05828daf7d5eecdea3}
O comando v coloca um estado em todas as grelhas de uma coluna


\begin{DoxyParams}{Parâmetros}
{\em A} & função coloca o estado de todas as grelhas da coluna n o num que ainda não estão determinadas como sendo água, ou seja, onde está '.' vai colocar '$\sim$' na coluna correspondente. \\
\hline
{\em c} & e l são as variáveis correspondentes à linha e coluna correspondentes.\\
\hline
\end{DoxyParams}
\begin{DoxyReturn}{Retorna}
Não retorna nada apenas altera o tabuleiro. 
\end{DoxyReturn}
\hypertarget{header_8h_afe6790c34e10764ba4ceaed8103a4e9d}{\index{header.\-h@{header.\-h}!cmd\-\_\-\-V@{cmd\-\_\-\-V}}
\index{cmd\-\_\-\-V@{cmd\-\_\-\-V}!header.h@{header.\-h}}
\subsubsection[{cmd\-\_\-\-V}]{\setlength{\rightskip}{0pt plus 5cm}int cmd\-\_\-\-V (
\begin{DoxyParamCaption}
\item[{{\bf E\-S\-T\-R\-U\-T\-U\-R\-A\-\_\-\-B\-N} $\ast$}]{est\-\_\-bn}
\end{DoxyParamCaption}
)}}\label{header_8h_afe6790c34e10764ba4ceaed8103a4e9d}
O comando V verifica se a solução apresentada está correta.


\begin{DoxyParams}{Parâmetros}
{\em A} & função é composta por várias funções auxiliares que conforme as regras da água, dos segmentos e dos barcos verifica se a solução apresentada está correta\\
\hline
\end{DoxyParams}
\begin{DoxyReturn}{Retorna}
A função retorna S\-I\-M caso a solução seja a correta e retorna N\-A\-O caso seja incorreta. 
\end{DoxyReturn}
\hypertarget{header_8h_a9e6501a1db374bad0edc521175045db2}{\index{header.\-h@{header.\-h}!contaaguas@{contaaguas}}
\index{contaaguas@{contaaguas}!header.h@{header.\-h}}
\subsubsection[{contaaguas}]{\setlength{\rightskip}{0pt plus 5cm}int contaaguas (
\begin{DoxyParamCaption}
\item[{{\bf E\-S\-T\-R\-U\-T\-U\-R\-A\-\_\-\-B\-N} $\ast$}]{}
\end{DoxyParamCaption}
)}}\label{header_8h_a9e6501a1db374bad0edc521175045db2}
\hypertarget{header_8h_ad1017c7d86f4155e2b7a54370a6d2505}{\index{header.\-h@{header.\-h}!contapc@{contapc}}
\index{contapc@{contapc}!header.h@{header.\-h}}
\subsubsection[{contapc}]{\setlength{\rightskip}{0pt plus 5cm}int contapc (
\begin{DoxyParamCaption}
\item[{{\bf E\-S\-T\-R\-U\-T\-U\-R\-A\-\_\-\-B\-N} $\ast$}]{, }
\item[{int}]{}
\end{DoxyParamCaption}
)}}\label{header_8h_ad1017c7d86f4155e2b7a54370a6d2505}
\hypertarget{header_8h_a4b82895cd2d62612c5f4d6b70cc38859}{\index{header.\-h@{header.\-h}!contapl@{contapl}}
\index{contapl@{contapl}!header.h@{header.\-h}}
\subsubsection[{contapl}]{\setlength{\rightskip}{0pt plus 5cm}int contapl (
\begin{DoxyParamCaption}
\item[{{\bf E\-S\-T\-R\-U\-T\-U\-R\-A\-\_\-\-B\-N} $\ast$}]{, }
\item[{int}]{}
\end{DoxyParamCaption}
)}}\label{header_8h_a4b82895cd2d62612c5f4d6b70cc38859}
\hypertarget{header_8h_ae5c776561dc50e0c46b28563ef8bba5d}{\index{header.\-h@{header.\-h}!contasegc@{contasegc}}
\index{contasegc@{contasegc}!header.h@{header.\-h}}
\subsubsection[{contasegc}]{\setlength{\rightskip}{0pt plus 5cm}int contasegc (
\begin{DoxyParamCaption}
\item[{{\bf E\-S\-T\-R\-U\-T\-U\-R\-A\-\_\-\-B\-N} $\ast$}]{, }
\item[{int}]{}
\end{DoxyParamCaption}
)}}\label{header_8h_ae5c776561dc50e0c46b28563ef8bba5d}
\hypertarget{header_8h_a75ba6d65db66767f9bdafe6ed1f59146}{\index{header.\-h@{header.\-h}!contasegl@{contasegl}}
\index{contasegl@{contasegl}!header.h@{header.\-h}}
\subsubsection[{contasegl}]{\setlength{\rightskip}{0pt plus 5cm}int contasegl (
\begin{DoxyParamCaption}
\item[{{\bf E\-S\-T\-R\-U\-T\-U\-R\-A\-\_\-\-B\-N} $\ast$}]{, }
\item[{int}]{}
\end{DoxyParamCaption}
)}}\label{header_8h_a75ba6d65db66767f9bdafe6ed1f59146}
\hypertarget{header_8h_a3891e7cfa8326204b890100cc4415901}{\index{header.\-h@{header.\-h}!E1\-\_\-filtraresto@{E1\-\_\-filtraresto}}
\index{E1\-\_\-filtraresto@{E1\-\_\-filtraresto}!header.h@{header.\-h}}
\subsubsection[{E1\-\_\-filtraresto}]{\setlength{\rightskip}{0pt plus 5cm}void E1\-\_\-filtraresto (
\begin{DoxyParamCaption}
\item[{{\bf E\-S\-T\-R\-U\-T\-U\-R\-A\-\_\-\-B\-N} $\ast$}]{}
\end{DoxyParamCaption}
)}}\label{header_8h_a3891e7cfa8326204b890100cc4415901}
\hypertarget{header_8h_aa2e5bd15a8e36136d2e93f7152d18b31}{\index{header.\-h@{header.\-h}!E1\-\_\-primlinha@{E1\-\_\-primlinha}}
\index{E1\-\_\-primlinha@{E1\-\_\-primlinha}!header.h@{header.\-h}}
\subsubsection[{E1\-\_\-primlinha}]{\setlength{\rightskip}{0pt plus 5cm}void E1\-\_\-primlinha (
\begin{DoxyParamCaption}
\item[{{\bf E\-S\-T\-R\-U\-T\-U\-R\-A\-\_\-\-B\-N} $\ast$}]{}
\end{DoxyParamCaption}
)}}\label{header_8h_aa2e5bd15a8e36136d2e93f7152d18b31}
\hypertarget{header_8h_a7c51831c11c69adde64f64bb7fd467dd}{\index{header.\-h@{header.\-h}!E2\-C@{E2\-C}}
\index{E2\-C@{E2\-C}!header.h@{header.\-h}}
\subsubsection[{E2\-C}]{\setlength{\rightskip}{0pt plus 5cm}int E2\-C (
\begin{DoxyParamCaption}
\item[{{\bf E\-S\-T\-R\-U\-T\-U\-R\-A\-\_\-\-B\-N} $\ast$}]{}
\end{DoxyParamCaption}
)}}\label{header_8h_a7c51831c11c69adde64f64bb7fd467dd}
\hypertarget{header_8h_a74de530d637c553d7713b08eb2ca7b74}{\index{header.\-h@{header.\-h}!E2\-L@{E2\-L}}
\index{E2\-L@{E2\-L}!header.h@{header.\-h}}
\subsubsection[{E2\-L}]{\setlength{\rightskip}{0pt plus 5cm}int E2\-L (
\begin{DoxyParamCaption}
\item[{{\bf E\-S\-T\-R\-U\-T\-U\-R\-A\-\_\-\-B\-N} $\ast$}]{}
\end{DoxyParamCaption}
)}}\label{header_8h_a74de530d637c553d7713b08eb2ca7b74}
\hypertarget{header_8h_aad6d3a3fe8e10625615605555c675715}{\index{header.\-h@{header.\-h}!haux@{haux}}
\index{haux@{haux}!header.h@{header.\-h}}
\subsubsection[{haux}]{\setlength{\rightskip}{0pt plus 5cm}void haux (
\begin{DoxyParamCaption}
\item[{{\bf E\-S\-T\-R\-U\-T\-U\-R\-A\-\_\-\-B\-N} $\ast$}]{, }
\item[{int}]{}
\end{DoxyParamCaption}
)}}\label{header_8h_aad6d3a3fe8e10625615605555c675715}
\hypertarget{header_8h_af08d169f9b572a9ac298559fc5b15cb2}{\index{header.\-h@{header.\-h}!interpretador@{interpretador}}
\index{interpretador@{interpretador}!header.h@{header.\-h}}
\subsubsection[{interpretador}]{\setlength{\rightskip}{0pt plus 5cm}void interpretador (
\begin{DoxyParamCaption}
\item[{{\bf E\-S\-T\-R\-U\-T\-U\-R\-A\-\_\-\-B\-N} $\ast$}]{est\-\_\-bn, }
\item[{{\bf stck} $\ast$}]{stack}
\end{DoxyParamCaption}
)}}\label{header_8h_af08d169f9b572a9ac298559fc5b15cb2}
O interpretador encarrega-\/se de criar um ciclo até que o comando seja igual N\-U\-L\-L \hypertarget{header_8h_a60c8bccf0a0a5948f9edf6c4ed1c145c}{\index{header.\-h@{header.\-h}!interpretar@{interpretar}}
\index{interpretar@{interpretar}!header.h@{header.\-h}}
\subsubsection[{interpretar}]{\setlength{\rightskip}{0pt plus 5cm}int interpretar (
\begin{DoxyParamCaption}
\item[{{\bf E\-S\-T\-R\-U\-T\-U\-R\-A\-\_\-\-B\-N} $\ast$}]{est\-\_\-bn, }
\item[{char $\ast$}]{linha, }
\item[{{\bf stck} $\ast$}]{stack}
\end{DoxyParamCaption}
)}}\label{header_8h_a60c8bccf0a0a5948f9edf6c4ed1c145c}
O comando interpretar está encarregue de compreender o comando inserido


\begin{DoxyParams}{Parâmetros}
{\em Conforme} & o comando inserido o interpretar tem um switch case para os diferentes comandos.\\
\hline
\end{DoxyParams}
\begin{DoxyReturn}{Retorna}
O resultado da função escrita. 
\end{DoxyReturn}
\hypertarget{header_8h_afed3fbcd069873008da84889cc81a19a}{\index{header.\-h@{header.\-h}!new\-Dados\-Est@{new\-Dados\-Est}}
\index{new\-Dados\-Est@{new\-Dados\-Est}!header.h@{header.\-h}}
\subsubsection[{new\-Dados\-Est}]{\setlength{\rightskip}{0pt plus 5cm}{\bf E\-S\-T\-R\-U\-T\-U\-R\-A\-\_\-\-B\-N}$\ast$ new\-Dados\-Est (
\begin{DoxyParamCaption}
\item[{{\bf E\-S\-T\-R\-U\-T\-U\-R\-A\-\_\-\-B\-N} $\ast$}]{est\-\_\-bn}
\end{DoxyParamCaption}
)}}\label{header_8h_afed3fbcd069873008da84889cc81a19a}
A função \char`\"{}new\-Dados\-Est\char`\"{} cria uma nova estrutura do tipo \char`\"{}\-E\-S\-T\-R\-U\-T\-U\-R\-A\-\_\-\-B\-N\char`\"{}.


\begin{DoxyParams}{Parâmetros}
{\em A} & função recebe como parametro somente uma estrutura, a qual vai copiar na integra.\\
\hline
\end{DoxyParams}
\begin{DoxyReturn}{Retorna}
A nova estrutura. 
\end{DoxyReturn}
\hypertarget{header_8h_a50a30835711378fc3a6669f18f18c1d4}{\index{header.\-h@{header.\-h}!new\-Dados\-P@{new\-Dados\-P}}
\index{new\-Dados\-P@{new\-Dados\-P}!header.h@{header.\-h}}
\subsubsection[{new\-Dados\-P}]{\setlength{\rightskip}{0pt plus 5cm}{\bf Dados\-P} $\ast$ new\-Dados\-P (
\begin{DoxyParamCaption}
\item[{int}]{lin, }
\item[{int}]{col, }
\item[{char}]{chr}
\end{DoxyParamCaption}
)}}\label{header_8h_a50a30835711378fc3a6669f18f18c1d4}
Esta funcao cria os dados relativos a uma unica posição.


\begin{DoxyParams}{Parâmetros}
{\em a} & funcao recebe os números da linha e coluna alterados. \\
\hline
{\em recebe} & também o carater que estava na posicao antes de esta ser alterada.\\
\hline
\end{DoxyParams}
\begin{DoxyReturn}{Retorna}
os dados relativos à posição. 
\end{DoxyReturn}
\hypertarget{header_8h_aa5babe6829adc82772460605370afade}{\index{header.\-h@{header.\-h}!new\-Dados\-V\-H@{new\-Dados\-V\-H}}
\index{new\-Dados\-V\-H@{new\-Dados\-V\-H}!header.h@{header.\-h}}
\subsubsection[{new\-Dados\-V\-H}]{\setlength{\rightskip}{0pt plus 5cm}{\bf Dados\-V\-H}$\ast$ new\-Dados\-V\-H (
\begin{DoxyParamCaption}
\item[{int}]{, }
\item[{int}]{, }
\item[{int}]{\mbox{[}$\,$\mbox{]}}
\end{DoxyParamCaption}
)}}\label{header_8h_aa5babe6829adc82772460605370afade}
\hypertarget{header_8h_aba53266076b1ba73d7dc0db01a58c9ee}{\index{header.\-h@{header.\-h}!pertence@{pertence}}
\index{pertence@{pertence}!header.h@{header.\-h}}
\subsubsection[{pertence}]{\setlength{\rightskip}{0pt plus 5cm}int pertence (
\begin{DoxyParamCaption}
\item[{char}]{}
\end{DoxyParamCaption}
)}}\label{header_8h_aba53266076b1ba73d7dc0db01a58c9ee}
\hypertarget{header_8h_a6e5986b8a99d54b0edd7f5d292cf9e7c}{\index{header.\-h@{header.\-h}!pertence3@{pertence3}}
\index{pertence3@{pertence3}!header.h@{header.\-h}}
\subsubsection[{pertence3}]{\setlength{\rightskip}{0pt plus 5cm}int pertence3 (
\begin{DoxyParamCaption}
\item[{char}]{}
\end{DoxyParamCaption}
)}}\label{header_8h_a6e5986b8a99d54b0edd7f5d292cf9e7c}
\hypertarget{header_8h_a4c3cc8ed850019e6ec428043c717a89f}{\index{header.\-h@{header.\-h}!poebarcos\-Auxc@{poebarcos\-Auxc}}
\index{poebarcos\-Auxc@{poebarcos\-Auxc}!header.h@{header.\-h}}
\subsubsection[{poebarcos\-Auxc}]{\setlength{\rightskip}{0pt plus 5cm}void poebarcos\-Auxc (
\begin{DoxyParamCaption}
\item[{{\bf E\-S\-T\-R\-U\-T\-U\-R\-A\-\_\-\-B\-N} $\ast$}]{, }
\item[{int}]{, }
\item[{int}]{}
\end{DoxyParamCaption}
)}}\label{header_8h_a4c3cc8ed850019e6ec428043c717a89f}
\hypertarget{header_8h_a6e802f4810aae2a030eb414236840023}{\index{header.\-h@{header.\-h}!poebarcos\-Auxl@{poebarcos\-Auxl}}
\index{poebarcos\-Auxl@{poebarcos\-Auxl}!header.h@{header.\-h}}
\subsubsection[{poebarcos\-Auxl}]{\setlength{\rightskip}{0pt plus 5cm}void poebarcos\-Auxl (
\begin{DoxyParamCaption}
\item[{{\bf E\-S\-T\-R\-U\-T\-U\-R\-A\-\_\-\-B\-N} $\ast$}]{, }
\item[{int}]{, }
\item[{int}]{}
\end{DoxyParamCaption}
)}}\label{header_8h_a6e802f4810aae2a030eb414236840023}
\hypertarget{header_8h_aabbb8b7ed90726e3bf98dd13d57b54b3}{\index{header.\-h@{header.\-h}!pop@{pop}}
\index{pop@{pop}!header.h@{header.\-h}}
\subsubsection[{pop}]{\setlength{\rightskip}{0pt plus 5cm}void pop (
\begin{DoxyParamCaption}
\item[{{\bf stck} $\ast$}]{stack}
\end{DoxyParamCaption}
)}}\label{header_8h_aabbb8b7ed90726e3bf98dd13d57b54b3}
Esta função retira um elemento da stack.


\begin{DoxyParams}{Parâmetros}
{\em esta} & função recebe somente a stack. \\
\hline
\end{DoxyParams}
\hypertarget{header_8h_a7163da75079918d15bb375d69af2dfd9}{\index{header.\-h@{header.\-h}!porbarcos@{porbarcos}}
\index{porbarcos@{porbarcos}!header.h@{header.\-h}}
\subsubsection[{porbarcos}]{\setlength{\rightskip}{0pt plus 5cm}int porbarcos (
\begin{DoxyParamCaption}
\item[{{\bf E\-S\-T\-R\-U\-T\-U\-R\-A\-\_\-\-B\-N} $\ast$}]{}
\end{DoxyParamCaption}
)}}\label{header_8h_a7163da75079918d15bb375d69af2dfd9}
\hypertarget{header_8h_a3751d683bcb249d0b7d1dcf65b6c5d97}{\index{header.\-h@{header.\-h}!poros@{poros}}
\index{poros@{poros}!header.h@{header.\-h}}
\subsubsection[{poros}]{\setlength{\rightskip}{0pt plus 5cm}void poros (
\begin{DoxyParamCaption}
\item[{{\bf E\-S\-T\-R\-U\-T\-U\-R\-A\-\_\-\-B\-N} $\ast$}]{}
\end{DoxyParamCaption}
)}}\label{header_8h_a3751d683bcb249d0b7d1dcf65b6c5d97}
\hypertarget{header_8h_a7990488aaeb3798cea2a7cb34f5f3945}{\index{header.\-h@{header.\-h}!porosc@{porosc}}
\index{porosc@{porosc}!header.h@{header.\-h}}
\subsubsection[{porosc}]{\setlength{\rightskip}{0pt plus 5cm}void porosc (
\begin{DoxyParamCaption}
\item[{{\bf E\-S\-T\-R\-U\-T\-U\-R\-A\-\_\-\-B\-N} $\ast$}]{}
\end{DoxyParamCaption}
)}}\label{header_8h_a7990488aaeb3798cea2a7cb34f5f3945}
\hypertarget{header_8h_a0d3e599ef4a848732e19b97da6696ae7}{\index{header.\-h@{header.\-h}!porosl@{porosl}}
\index{porosl@{porosl}!header.h@{header.\-h}}
\subsubsection[{porosl}]{\setlength{\rightskip}{0pt plus 5cm}void porosl (
\begin{DoxyParamCaption}
\item[{{\bf E\-S\-T\-R\-U\-T\-U\-R\-A\-\_\-\-B\-N} $\ast$}]{}
\end{DoxyParamCaption}
)}}\label{header_8h_a0d3e599ef4a848732e19b97da6696ae7}
\hypertarget{header_8h_aae86d3206ebffbec52cd0050e79d96ee}{\index{header.\-h@{header.\-h}!porsegs@{porsegs}}
\index{porsegs@{porsegs}!header.h@{header.\-h}}
\subsubsection[{porsegs}]{\setlength{\rightskip}{0pt plus 5cm}int porsegs (
\begin{DoxyParamCaption}
\item[{{\bf E\-S\-T\-R\-U\-T\-U\-R\-A\-\_\-\-B\-N} $\ast$}]{est\-\_\-bn}
\end{DoxyParamCaption}
)}}\label{header_8h_aae86d3206ebffbec52cd0050e79d96ee}
Esta função substitui os o's que sejam possiveis substituir por segmentos de barcos.


\begin{DoxyParams}{Parâmetros}
{\em a} & função recebe a estrutura com todos os o's.\\
\hline
\end{DoxyParams}
\begin{DoxyReturn}{Retorna}
1 ou zero consoante haja alterações 
\end{DoxyReturn}
\hypertarget{header_8h_aa7af588d3c0710101b7fd4563e564982}{\index{header.\-h@{header.\-h}!porsubs@{porsubs}}
\index{porsubs@{porsubs}!header.h@{header.\-h}}
\subsubsection[{porsubs}]{\setlength{\rightskip}{0pt plus 5cm}int porsubs (
\begin{DoxyParamCaption}
\item[{{\bf E\-S\-T\-R\-U\-T\-U\-R\-A\-\_\-\-B\-N} $\ast$}]{}
\end{DoxyParamCaption}
)}}\label{header_8h_aa7af588d3c0710101b7fd4563e564982}
\hypertarget{header_8h_ab388a875e4fa77316ae653ff71a6214e}{\index{header.\-h@{header.\-h}!push@{push}}
\index{push@{push}!header.h@{header.\-h}}
\subsubsection[{push}]{\setlength{\rightskip}{0pt plus 5cm}void push (
\begin{DoxyParamCaption}
\item[{{\bf stck} $\ast$}]{stack, }
\item[{char}]{cmd, }
\item[{union dados}]{d}
\end{DoxyParamCaption}
)}}\label{header_8h_ab388a875e4fa77316ae653ff71a6214e}
Esta funcao acrescenta um elemento à stack


\begin{DoxyParams}{Parâmetros}
{\em a} & função recebe a stack. \\
\hline
{\em recebe} & o comando que foi utilizado. \\
\hline
{\em recebe} & uma \char`\"{}union dados\char`\"{} que, dependendo do comando uzado será um apontador para tipos diferentes. \\
\hline
\end{DoxyParams}
\hypertarget{header_8h_a0858462671024466f8669aed58a00d81}{\index{header.\-h@{header.\-h}!tiraros@{tiraros}}
\index{tiraros@{tiraros}!header.h@{header.\-h}}
\subsubsection[{tiraros}]{\setlength{\rightskip}{0pt plus 5cm}void tiraros (
\begin{DoxyParamCaption}
\item[{{\bf E\-S\-T\-R\-U\-T\-U\-R\-A\-\_\-\-B\-N} $\ast$}]{}
\end{DoxyParamCaption}
)}}\label{header_8h_a0858462671024466f8669aed58a00d81}
\hypertarget{header_8h_a512a6ef1c8fef57ded24e2f7adb5347c}{\index{header.\-h@{header.\-h}!undo\-\_\-\-C\-L\-E@{undo\-\_\-\-C\-L\-E}}
\index{undo\-\_\-\-C\-L\-E@{undo\-\_\-\-C\-L\-E}!header.h@{header.\-h}}
\subsubsection[{undo\-\_\-\-C\-L\-E}]{\setlength{\rightskip}{0pt plus 5cm}void undo\-\_\-\-C\-L\-E (
\begin{DoxyParamCaption}
\item[{{\bf E\-S\-T\-R\-U\-T\-U\-R\-A\-\_\-\-B\-N} $\ast$}]{est\-\_\-bn, }
\item[{{\bf stck} $\ast$}]{stack}
\end{DoxyParamCaption}
)}}\label{header_8h_a512a6ef1c8fef57ded24e2f7adb5347c}
Esta função está responsável por desfazer o último comando quando este foi o C, o L, ou qualquer das E's.


\begin{DoxyParams}{Parâmetros}
{\em a} & função recebe a estrutura atual \\
\hline
{\em para} & que possa recuperar a estrutura antiga, a função recebe também a stack. \\
\hline
\end{DoxyParams}
\hypertarget{header_8h_ae62b88899707025f3893fd2939b4e3de}{\index{header.\-h@{header.\-h}!undo\-\_\-h@{undo\-\_\-h}}
\index{undo\-\_\-h@{undo\-\_\-h}!header.h@{header.\-h}}
\subsubsection[{undo\-\_\-h}]{\setlength{\rightskip}{0pt plus 5cm}void undo\-\_\-h (
\begin{DoxyParamCaption}
\item[{{\bf E\-S\-T\-R\-U\-T\-U\-R\-A\-\_\-\-B\-N} $\ast$}]{est\-\_\-bn, }
\item[{{\bf stck} $\ast$}]{stack}
\end{DoxyParamCaption}
)}}\label{header_8h_ae62b88899707025f3893fd2939b4e3de}
Esta função está responsável por desfazer o último comando quando este foi o h.


\begin{DoxyParams}{Parâmetros}
{\em a} & função recebe a estrutura atual \\
\hline
{\em para} & que possa recuperar a estrutura antiga, a função recebe também a stack. \\
\hline
\end{DoxyParams}
\hypertarget{header_8h_aebfff5a330995069c917ea559880f1cc}{\index{header.\-h@{header.\-h}!undo\-\_\-p@{undo\-\_\-p}}
\index{undo\-\_\-p@{undo\-\_\-p}!header.h@{header.\-h}}
\subsubsection[{undo\-\_\-p}]{\setlength{\rightskip}{0pt plus 5cm}void undo\-\_\-p (
\begin{DoxyParamCaption}
\item[{{\bf E\-S\-T\-R\-U\-T\-U\-R\-A\-\_\-\-B\-N} $\ast$}]{est\-\_\-bn, }
\item[{{\bf stck} $\ast$}]{stack}
\end{DoxyParamCaption}
)}}\label{header_8h_aebfff5a330995069c917ea559880f1cc}
Esta função está responsável por desfazer o último comando quando este foi o p.


\begin{DoxyParams}{Parâmetros}
{\em a} & função recebe a estrutura atual \\
\hline
{\em para} & que possa recuperar a estrutura antiga, a função recebe também a stack. \\
\hline
\end{DoxyParams}
\hypertarget{header_8h_a3eb15c46ca4a0bbf5f8583106e31443a}{\index{header.\-h@{header.\-h}!undo\-\_\-v@{undo\-\_\-v}}
\index{undo\-\_\-v@{undo\-\_\-v}!header.h@{header.\-h}}
\subsubsection[{undo\-\_\-v}]{\setlength{\rightskip}{0pt plus 5cm}void undo\-\_\-v (
\begin{DoxyParamCaption}
\item[{{\bf E\-S\-T\-R\-U\-T\-U\-R\-A\-\_\-\-B\-N} $\ast$}]{est\-\_\-bn, }
\item[{{\bf stck} $\ast$}]{stack}
\end{DoxyParamCaption}
)}}\label{header_8h_a3eb15c46ca4a0bbf5f8583106e31443a}
Esta função está responsável por desfazer o último comando quando este foi o v.


\begin{DoxyParams}{Parâmetros}
{\em a} & função recebe a estrutura atual \\
\hline
{\em para} & que possa recuperar a estrutura antiga, a função recebe também a stack. \\
\hline
\end{DoxyParams}
\hypertarget{header_8h_ad51e83dedad9fe2e52a2842db30b3434}{\index{header.\-h@{header.\-h}!vaux@{vaux}}
\index{vaux@{vaux}!header.h@{header.\-h}}
\subsubsection[{vaux}]{\setlength{\rightskip}{0pt plus 5cm}void vaux (
\begin{DoxyParamCaption}
\item[{{\bf E\-S\-T\-R\-U\-T\-U\-R\-A\-\_\-\-B\-N} $\ast$}]{, }
\item[{int}]{}
\end{DoxyParamCaption}
)}}\label{header_8h_ad51e83dedad9fe2e52a2842db30b3434}
\hypertarget{header_8h_a5fb6f50cbc25a556b19bc095533f6393}{\index{header.\-h@{header.\-h}!verifica\-Agua@{verifica\-Agua}}
\index{verifica\-Agua@{verifica\-Agua}!header.h@{header.\-h}}
\subsubsection[{verifica\-Agua}]{\setlength{\rightskip}{0pt plus 5cm}int verifica\-Agua (
\begin{DoxyParamCaption}
\item[{{\bf E\-S\-T\-R\-U\-T\-U\-R\-A\-\_\-\-B\-N} $\ast$}]{}
\end{DoxyParamCaption}
)}}\label{header_8h_a5fb6f50cbc25a556b19bc095533f6393}
\hypertarget{header_8h_a172638f19aff4f9bd3a018a16b5f98e8}{\index{header.\-h@{header.\-h}!verifica\-Barcos@{verifica\-Barcos}}
\index{verifica\-Barcos@{verifica\-Barcos}!header.h@{header.\-h}}
\subsubsection[{verifica\-Barcos}]{\setlength{\rightskip}{0pt plus 5cm}int verifica\-Barcos (
\begin{DoxyParamCaption}
\item[{{\bf E\-S\-T\-R\-U\-T\-U\-R\-A\-\_\-\-B\-N} $\ast$}]{}
\end{DoxyParamCaption}
)}}\label{header_8h_a172638f19aff4f9bd3a018a16b5f98e8}
\hypertarget{header_8h_a3f4806bc2a02a5a364967823141b10ac}{\index{header.\-h@{header.\-h}!verifica\-Cantos\-P@{verifica\-Cantos\-P}}
\index{verifica\-Cantos\-P@{verifica\-Cantos\-P}!header.h@{header.\-h}}
\subsubsection[{verifica\-Cantos\-P}]{\setlength{\rightskip}{0pt plus 5cm}int verifica\-Cantos\-P (
\begin{DoxyParamCaption}
\item[{{\bf E\-S\-T\-R\-U\-T\-U\-R\-A\-\_\-\-B\-N} $\ast$}]{}
\end{DoxyParamCaption}
)}}\label{header_8h_a3f4806bc2a02a5a364967823141b10ac}
\hypertarget{header_8h_ac3996c0889a577d3df4d672fbd4d5b95}{\index{header.\-h@{header.\-h}!verifica\-Peca\-Ex@{verifica\-Peca\-Ex}}
\index{verifica\-Peca\-Ex@{verifica\-Peca\-Ex}!header.h@{header.\-h}}
\subsubsection[{verifica\-Peca\-Ex}]{\setlength{\rightskip}{0pt plus 5cm}int verifica\-Peca\-Ex (
\begin{DoxyParamCaption}
\item[{{\bf E\-S\-T\-R\-U\-T\-U\-R\-A\-\_\-\-B\-N} $\ast$}]{}
\end{DoxyParamCaption}
)}}\label{header_8h_ac3996c0889a577d3df4d672fbd4d5b95}
\hypertarget{header_8h_a84da85ba6c2068180f8c976bd4ccbea5}{\index{header.\-h@{header.\-h}!verifica\-Seg@{verifica\-Seg}}
\index{verifica\-Seg@{verifica\-Seg}!header.h@{header.\-h}}
\subsubsection[{verifica\-Seg}]{\setlength{\rightskip}{0pt plus 5cm}int verifica\-Seg (
\begin{DoxyParamCaption}
\item[{{\bf E\-S\-T\-R\-U\-T\-U\-R\-A\-\_\-\-B\-N} $\ast$}]{}
\end{DoxyParamCaption}
)}}\label{header_8h_a84da85ba6c2068180f8c976bd4ccbea5}
\hypertarget{header_8h_a8ab3436eacf68a0f7a7a31f2c7d13397}{\index{header.\-h@{header.\-h}!verifica\-S\-U\-B@{verifica\-S\-U\-B}}
\index{verifica\-S\-U\-B@{verifica\-S\-U\-B}!header.h@{header.\-h}}
\subsubsection[{verifica\-S\-U\-B}]{\setlength{\rightskip}{0pt plus 5cm}int verifica\-S\-U\-B (
\begin{DoxyParamCaption}
\item[{{\bf E\-S\-T\-R\-U\-T\-U\-R\-A\-\_\-\-B\-N} $\ast$}]{, }
\item[{char}]{, }
\item[{int}]{, }
\item[{int}]{}
\end{DoxyParamCaption}
)}}\label{header_8h_a8ab3436eacf68a0f7a7a31f2c7d13397}

\hypertarget{interpretador_8c}{\section{Referência ao ficheiro interpretador.\-c}
\label{interpretador_8c}\index{interpretador.\-c@{interpretador.\-c}}
}
{\ttfamily \#include $<$stdio.\-h$>$}\\*
{\ttfamily \#include $<$stdlib.\-h$>$}\\*
{\ttfamily \#include $<$string.\-h$>$}\\*
{\ttfamily \#include \char`\"{}header.\-h\char`\"{}}\\*
\subsection*{Funções}
\begin{DoxyCompactItemize}
\item 
int \hyperlink{interpretador_8c_ac8c1b2b92769d39a81df16d46072f51f}{interpretar} (\hyperlink{header_8h_a341ac1667f3dc23635b071398592e724}{E\-S\-T\-R\-U\-T\-U\-R\-A\-\_\-\-B\-N} $\ast$\hyperlink{structest__bn}{est\-\_\-bn}, char $\ast$linha, \hyperlink{header_8h_a6209c6f0cbf77a7146be05f6df115ceb}{stck} $\ast$\hyperlink{structstack}{stack})
\item 
void \hyperlink{interpretador_8c_a46134b4f7006ed872c2c6acecf5f0cee}{interpretador} (\hyperlink{header_8h_a341ac1667f3dc23635b071398592e724}{E\-S\-T\-R\-U\-T\-U\-R\-A\-\_\-\-B\-N} $\ast$\hyperlink{structest__bn}{est\-\_\-bn}, \hyperlink{header_8h_a6209c6f0cbf77a7146be05f6df115ceb}{stck} $\ast$\hyperlink{structstack}{stack})
\item 
int \hyperlink{interpretador_8c_ae66f6b31b5ad750f1fe042a706a4e3d4}{main} ()
\end{DoxyCompactItemize}


\subsection{Documentação das funções}
\hypertarget{interpretador_8c_a46134b4f7006ed872c2c6acecf5f0cee}{\index{interpretador.\-c@{interpretador.\-c}!interpretador@{interpretador}}
\index{interpretador@{interpretador}!interpretador.c@{interpretador.\-c}}
\subsubsection[{interpretador}]{\setlength{\rightskip}{0pt plus 5cm}void interpretador (
\begin{DoxyParamCaption}
\item[{{\bf E\-S\-T\-R\-U\-T\-U\-R\-A\-\_\-\-B\-N} $\ast$}]{est\-\_\-bn, }
\item[{{\bf stck} $\ast$}]{stack}
\end{DoxyParamCaption}
)}}\label{interpretador_8c_a46134b4f7006ed872c2c6acecf5f0cee}
O interpretador encarrega-\/se de criar um ciclo até que o comando seja igual N\-U\-L\-L \hypertarget{interpretador_8c_ac8c1b2b92769d39a81df16d46072f51f}{\index{interpretador.\-c@{interpretador.\-c}!interpretar@{interpretar}}
\index{interpretar@{interpretar}!interpretador.c@{interpretador.\-c}}
\subsubsection[{interpretar}]{\setlength{\rightskip}{0pt plus 5cm}int interpretar (
\begin{DoxyParamCaption}
\item[{{\bf E\-S\-T\-R\-U\-T\-U\-R\-A\-\_\-\-B\-N} $\ast$}]{est\-\_\-bn, }
\item[{char $\ast$}]{linha, }
\item[{{\bf stck} $\ast$}]{stack}
\end{DoxyParamCaption}
)}}\label{interpretador_8c_ac8c1b2b92769d39a81df16d46072f51f}
O comando interpretar está encarregue de compreender o comando inserido


\begin{DoxyParams}{Parâmetros}
{\em Conforme} & o comando inserido o interpretar tem um switch case para os diferentes comandos.\\
\hline
\end{DoxyParams}
\begin{DoxyReturn}{Retorna}
O resultado da função escrita. 
\end{DoxyReturn}
\hypertarget{interpretador_8c_ae66f6b31b5ad750f1fe042a706a4e3d4}{\index{interpretador.\-c@{interpretador.\-c}!main@{main}}
\index{main@{main}!interpretador.c@{interpretador.\-c}}
\subsubsection[{main}]{\setlength{\rightskip}{0pt plus 5cm}int main (
\begin{DoxyParamCaption}
{}
\end{DoxyParamCaption}
)}}\label{interpretador_8c_ae66f6b31b5ad750f1fe042a706a4e3d4}

\hypertarget{parte1_8c}{\section{Referência ao ficheiro parte1.\-c}
\label{parte1_8c}\index{parte1.\-c@{parte1.\-c}}
}
{\ttfamily \#include $<$stdio.\-h$>$}\\*
{\ttfamily \#include $<$string.\-h$>$}\\*
{\ttfamily \#include $<$stdlib.\-h$>$}\\*
{\ttfamily \#include \char`\"{}header.\-h\char`\"{}}\\*
\subsection*{Funções}
\begin{DoxyCompactItemize}
\item 
\hyperlink{header_8h_a341ac1667f3dc23635b071398592e724}{E\-S\-T\-R\-U\-T\-U\-R\-A\-\_\-\-B\-N} $\ast$ \hyperlink{parte1_8c_aef94537ea16678ff43e6c34ad7ff3505}{new\-Dados\-Est} (\hyperlink{header_8h_a341ac1667f3dc23635b071398592e724}{E\-S\-T\-R\-U\-T\-U\-R\-A\-\_\-\-B\-N} $\ast$\hyperlink{structest__bn}{est\-\_\-bn})
\item 
int \hyperlink{parte1_8c_ab72fb307b21be09f1012820f489a4f6f}{cmd\-\_\-c} (\hyperlink{header_8h_a341ac1667f3dc23635b071398592e724}{E\-S\-T\-R\-U\-T\-U\-R\-A\-\_\-\-B\-N} $\ast$\hyperlink{structest__bn}{est\-\_\-bn}, \hyperlink{header_8h_a6209c6f0cbf77a7146be05f6df115ceb}{stck} $\ast$\hyperlink{structstack}{stack})
\item 
int \hyperlink{parte1_8c_a1477aa7a5bfd07a0f88315d84a68cce2}{cmd\-\_\-m} (\hyperlink{header_8h_a341ac1667f3dc23635b071398592e724}{E\-S\-T\-R\-U\-T\-U\-R\-A\-\_\-\-B\-N} $\ast$\hyperlink{structest__bn}{est\-\_\-bn})
\item 
int \hyperlink{parte1_8c_ae590e32890755d1eb50586e0fa1747de}{cmd\-\_\-h} (\hyperlink{header_8h_a341ac1667f3dc23635b071398592e724}{E\-S\-T\-R\-U\-T\-U\-R\-A\-\_\-\-B\-N} $\ast$\hyperlink{structest__bn}{est\-\_\-bn}, int l, \hyperlink{header_8h_a6209c6f0cbf77a7146be05f6df115ceb}{stck} $\ast$\hyperlink{structstack}{stack})
\item 
\hyperlink{header_8h_ac31cca6d54fd1fce82eca72192564cd4}{Dados\-V\-H} $\ast$ \hyperlink{parte1_8c_a7acc435844c7e6fd7a371c9bac0503aa}{new\-Dados\-V\-H} (int lc, int t, int a\mbox{[}\hyperlink{header_8h_a0592dba56693fad79136250c11e5a7fe}{M\-A\-X\-\_\-\-S\-I\-Z\-E}\mbox{]})
\item 
int \hyperlink{parte1_8c_a8a7b914ecacb9b94a8d46cc3e36bb97f}{cmd\-\_\-v} (\hyperlink{header_8h_a341ac1667f3dc23635b071398592e724}{E\-S\-T\-R\-U\-T\-U\-R\-A\-\_\-\-B\-N} $\ast$\hyperlink{structest__bn}{est\-\_\-bn}, int c, \hyperlink{header_8h_a6209c6f0cbf77a7146be05f6df115ceb}{stck} $\ast$\hyperlink{structstack}{stack})
\item 
\hyperlink{header_8h_a469ff72408bc348b417cc0ee2cf860b0}{Dados\-P} $\ast$ \hyperlink{parte1_8c_aadd63d46c30af5d8abe03eca62515531}{new\-Dados\-P} (int lin, int col, char chr)
\item 
int \hyperlink{parte1_8c_aeb48731b2c91b25b35ee97d50446c1ae}{cmd\-\_\-p} (\hyperlink{header_8h_a341ac1667f3dc23635b071398592e724}{E\-S\-T\-R\-U\-T\-U\-R\-A\-\_\-\-B\-N} $\ast$\hyperlink{structest__bn}{est\-\_\-bn}, char x, int l, int c, \hyperlink{header_8h_a6209c6f0cbf77a7146be05f6df115ceb}{stck} $\ast$\hyperlink{structstack}{stack})
\end{DoxyCompactItemize}


\subsection{Documentação das funções}
\hypertarget{parte1_8c_ab72fb307b21be09f1012820f489a4f6f}{\index{parte1.\-c@{parte1.\-c}!cmd\-\_\-c@{cmd\-\_\-c}}
\index{cmd\-\_\-c@{cmd\-\_\-c}!parte1.c@{parte1.\-c}}
\subsubsection[{cmd\-\_\-c}]{\setlength{\rightskip}{0pt plus 5cm}int cmd\-\_\-c (
\begin{DoxyParamCaption}
\item[{{\bf E\-S\-T\-R\-U\-T\-U\-R\-A\-\_\-\-B\-N} $\ast$}]{est\-\_\-bn, }
\item[{{\bf stck} $\ast$}]{stack}
\end{DoxyParamCaption}
)}}\label{parte1_8c_ab72fb307b21be09f1012820f489a4f6f}
O comando c lê o tabuleiro a partir do standard input.


\begin{DoxyParams}{Parâmetros}
{\em A} & função vai receber um tabuleiro a partir do standard input. \\
\hline
{\em Recebe} & a estrutura (\hyperlink{structest__bn}{est\-\_\-bn}), dois arrays um linha (recebe a linha do tabuleiro) e restolinha (guarda o resto da linha). \\
\hline
{\em lin} & e col são as duas variáveis para o tamanho das linhas e colunas respetivamente.\\
\hline
\end{DoxyParams}
\begin{DoxyReturn}{Retorna}
Não retorna nada apenas lê o tabuleiro a partir do teclado. 
\end{DoxyReturn}
\hypertarget{parte1_8c_ae590e32890755d1eb50586e0fa1747de}{\index{parte1.\-c@{parte1.\-c}!cmd\-\_\-h@{cmd\-\_\-h}}
\index{cmd\-\_\-h@{cmd\-\_\-h}!parte1.c@{parte1.\-c}}
\subsubsection[{cmd\-\_\-h}]{\setlength{\rightskip}{0pt plus 5cm}int cmd\-\_\-h (
\begin{DoxyParamCaption}
\item[{{\bf E\-S\-T\-R\-U\-T\-U\-R\-A\-\_\-\-B\-N} $\ast$}]{est\-\_\-bn, }
\item[{int}]{l, }
\item[{{\bf stck} $\ast$}]{stack}
\end{DoxyParamCaption}
)}}\label{parte1_8c_ae590e32890755d1eb50586e0fa1747de}
O comando h coloca um estado em todas as grelhas de uma linha.


\begin{DoxyParams}{Parâmetros}
{\em A} & função coloca o estado de todas as grelhas da linha n o num que ainda não estão determinadas como sendo água. \\
\hline
{\em Onde} & está '.' vai colocar '$\sim$' na linha correspondente.\\
\hline
\end{DoxyParams}
\begin{DoxyReturn}{Retorna}
Não retorna nada apenas altera o tabuleiro. 
\end{DoxyReturn}
\hypertarget{parte1_8c_a1477aa7a5bfd07a0f88315d84a68cce2}{\index{parte1.\-c@{parte1.\-c}!cmd\-\_\-m@{cmd\-\_\-m}}
\index{cmd\-\_\-m@{cmd\-\_\-m}!parte1.c@{parte1.\-c}}
\subsubsection[{cmd\-\_\-m}]{\setlength{\rightskip}{0pt plus 5cm}int cmd\-\_\-m (
\begin{DoxyParamCaption}
\item[{{\bf E\-S\-T\-R\-U\-T\-U\-R\-A\-\_\-\-B\-N} $\ast$}]{est\-\_\-bn}
\end{DoxyParamCaption}
)}}\label{parte1_8c_a1477aa7a5bfd07a0f88315d84a68cce2}
O comando m mostra o tabuleiro.


\begin{DoxyParams}{Parâmetros}
{\em A} & função vai receber um tabuleiro e vai imprimi-\/lo no ecrã. \\
\hline
{\em lin} & e col são as duas variáveis para o tamanho das linhas e colunas respetivamente.\\
\hline
\end{DoxyParams}
\begin{DoxyReturn}{Retorna}
Não retorna nada apenas mostra tabuleiro. 
\end{DoxyReturn}
\hypertarget{parte1_8c_aeb48731b2c91b25b35ee97d50446c1ae}{\index{parte1.\-c@{parte1.\-c}!cmd\-\_\-p@{cmd\-\_\-p}}
\index{cmd\-\_\-p@{cmd\-\_\-p}!parte1.c@{parte1.\-c}}
\subsubsection[{cmd\-\_\-p}]{\setlength{\rightskip}{0pt plus 5cm}int cmd\-\_\-p (
\begin{DoxyParamCaption}
\item[{{\bf E\-S\-T\-R\-U\-T\-U\-R\-A\-\_\-\-B\-N} $\ast$}]{est\-\_\-bn, }
\item[{char}]{x, }
\item[{int}]{l, }
\item[{int}]{c, }
\item[{{\bf stck} $\ast$}]{stack}
\end{DoxyParamCaption}
)}}\label{parte1_8c_aeb48731b2c91b25b35ee97d50446c1ae}
O comando p coloca uma peça de um barco ou submarino na linha e coluna dados.


\begin{DoxyParams}{Parâmetros}
{\em A} & função ao receber um carater deve colocar na linha e coluna correspondentes esse mesmo carater. \\
\hline
{\em ln} & e cl são as variáveis que correspondem à linha ln e coluna cl.\\
\hline
\end{DoxyParams}
\begin{DoxyReturn}{Retorna}
Não retorna nada apenas altera o tabuleiro. 
\end{DoxyReturn}
\hypertarget{parte1_8c_a8a7b914ecacb9b94a8d46cc3e36bb97f}{\index{parte1.\-c@{parte1.\-c}!cmd\-\_\-v@{cmd\-\_\-v}}
\index{cmd\-\_\-v@{cmd\-\_\-v}!parte1.c@{parte1.\-c}}
\subsubsection[{cmd\-\_\-v}]{\setlength{\rightskip}{0pt plus 5cm}int cmd\-\_\-v (
\begin{DoxyParamCaption}
\item[{{\bf E\-S\-T\-R\-U\-T\-U\-R\-A\-\_\-\-B\-N} $\ast$}]{est\-\_\-bn, }
\item[{int}]{c, }
\item[{{\bf stck} $\ast$}]{stack}
\end{DoxyParamCaption}
)}}\label{parte1_8c_a8a7b914ecacb9b94a8d46cc3e36bb97f}
O comando v coloca um estado em todas as grelhas de uma coluna


\begin{DoxyParams}{Parâmetros}
{\em A} & função coloca o estado de todas as grelhas da coluna n o num que ainda não estão determinadas como sendo água, ou seja, onde está '.' vai colocar '$\sim$' na coluna correspondente. \\
\hline
{\em c} & e l são as variáveis correspondentes à linha e coluna correspondentes.\\
\hline
\end{DoxyParams}
\begin{DoxyReturn}{Retorna}
Não retorna nada apenas altera o tabuleiro. 
\end{DoxyReturn}
\hypertarget{parte1_8c_aef94537ea16678ff43e6c34ad7ff3505}{\index{parte1.\-c@{parte1.\-c}!new\-Dados\-Est@{new\-Dados\-Est}}
\index{new\-Dados\-Est@{new\-Dados\-Est}!parte1.c@{parte1.\-c}}
\subsubsection[{new\-Dados\-Est}]{\setlength{\rightskip}{0pt plus 5cm}{\bf E\-S\-T\-R\-U\-T\-U\-R\-A\-\_\-\-B\-N}$\ast$ new\-Dados\-Est (
\begin{DoxyParamCaption}
\item[{{\bf E\-S\-T\-R\-U\-T\-U\-R\-A\-\_\-\-B\-N} $\ast$}]{est\-\_\-bn}
\end{DoxyParamCaption}
)}}\label{parte1_8c_aef94537ea16678ff43e6c34ad7ff3505}
A função \char`\"{}new\-Dados\-Est\char`\"{} cria uma nova estrutura do tipo \char`\"{}\-E\-S\-T\-R\-U\-T\-U\-R\-A\-\_\-\-B\-N\char`\"{}.


\begin{DoxyParams}{Parâmetros}
{\em A} & função recebe como parametro somente uma estrutura, a qual vai copiar na integra.\\
\hline
\end{DoxyParams}
\begin{DoxyReturn}{Retorna}
A nova estrutura. 
\end{DoxyReturn}
\hypertarget{parte1_8c_aadd63d46c30af5d8abe03eca62515531}{\index{parte1.\-c@{parte1.\-c}!new\-Dados\-P@{new\-Dados\-P}}
\index{new\-Dados\-P@{new\-Dados\-P}!parte1.c@{parte1.\-c}}
\subsubsection[{new\-Dados\-P}]{\setlength{\rightskip}{0pt plus 5cm}{\bf Dados\-P}$\ast$ new\-Dados\-P (
\begin{DoxyParamCaption}
\item[{int}]{lin, }
\item[{int}]{col, }
\item[{char}]{chr}
\end{DoxyParamCaption}
)}}\label{parte1_8c_aadd63d46c30af5d8abe03eca62515531}
Esta funcao cria os dados relativos a uma unica posição.


\begin{DoxyParams}{Parâmetros}
{\em a} & funcao recebe os números da linha e coluna alterados. \\
\hline
{\em recebe} & também o carater que estava na posicao antes de esta ser alterada.\\
\hline
\end{DoxyParams}
\begin{DoxyReturn}{Retorna}
os dados relativos à posição. 
\end{DoxyReturn}
\hypertarget{parte1_8c_a7acc435844c7e6fd7a371c9bac0503aa}{\index{parte1.\-c@{parte1.\-c}!new\-Dados\-V\-H@{new\-Dados\-V\-H}}
\index{new\-Dados\-V\-H@{new\-Dados\-V\-H}!parte1.c@{parte1.\-c}}
\subsubsection[{new\-Dados\-V\-H}]{\setlength{\rightskip}{0pt plus 5cm}{\bf Dados\-V\-H}$\ast$ new\-Dados\-V\-H (
\begin{DoxyParamCaption}
\item[{int}]{lc, }
\item[{int}]{t, }
\item[{int}]{a\mbox{[}\-M\-A\-X\-\_\-\-S\-I\-Z\-E\mbox{]}}
\end{DoxyParamCaption}
)}}\label{parte1_8c_a7acc435844c7e6fd7a371c9bac0503aa}
Esta função irá criar uma nova estrutur que guarda os dados de uma fila/coluna inteira.


\begin{DoxyParams}{Parâmetros}
{\em como} & primeiro argumento a função recebe o numero da linha/coluna que quer copiar. \\
\hline
{\em recebe} & também o tamanho da linha/coluna. \\
\hline
{\em por} & último a função recebe a propria linha/coluna que quer copiar.\\
\hline
\end{DoxyParams}
\begin{DoxyReturn}{Retorna}
os dados da fila ou coluna que foram copiados. 
\end{DoxyReturn}

\hypertarget{parte2_8c}{\section{Referência ao ficheiro parte2.\-c}
\label{parte2_8c}\index{parte2.\-c@{parte2.\-c}}
}
{\ttfamily \#include $<$stdio.\-h$>$}\\*
{\ttfamily \#include $<$string.\-h$>$}\\*
{\ttfamily \#include $<$stdlib.\-h$>$}\\*
{\ttfamily \#include \char`\"{}header.\-h\char`\"{}}\\*
\subsection*{Funções}
\begin{DoxyCompactItemize}
\item 
void \hyperlink{parte2_8c_a3612c739cc369c5547480cc53ca1b2fa}{undo\-\_\-p} (\hyperlink{header_8h_a341ac1667f3dc23635b071398592e724}{E\-S\-T\-R\-U\-T\-U\-R\-A\-\_\-\-B\-N} $\ast$\hyperlink{structest__bn}{est\-\_\-bn}, \hyperlink{header_8h_a6209c6f0cbf77a7146be05f6df115ceb}{stck} $\ast$\hyperlink{structstack}{stack})
\item 
void \hyperlink{parte2_8c_a530f2491dbaf2effb0937ca4fb964dfd}{undo\-\_\-v} (\hyperlink{header_8h_a341ac1667f3dc23635b071398592e724}{E\-S\-T\-R\-U\-T\-U\-R\-A\-\_\-\-B\-N} $\ast$\hyperlink{structest__bn}{est\-\_\-bn}, \hyperlink{header_8h_a6209c6f0cbf77a7146be05f6df115ceb}{stck} $\ast$\hyperlink{structstack}{stack})
\item 
void \hyperlink{parte2_8c_adda53bc08b6e1006a4a7168a7ac02d38}{undo\-\_\-h} (\hyperlink{header_8h_a341ac1667f3dc23635b071398592e724}{E\-S\-T\-R\-U\-T\-U\-R\-A\-\_\-\-B\-N} $\ast$\hyperlink{structest__bn}{est\-\_\-bn}, \hyperlink{header_8h_a6209c6f0cbf77a7146be05f6df115ceb}{stck} $\ast$\hyperlink{structstack}{stack})
\item 
void \hyperlink{parte2_8c_addea9cb6aecf35f686a0a33ff35456fa}{undo\-\_\-\-C\-L\-E} (\hyperlink{header_8h_a341ac1667f3dc23635b071398592e724}{E\-S\-T\-R\-U\-T\-U\-R\-A\-\_\-\-B\-N} $\ast$\hyperlink{structest__bn}{est\-\_\-bn}, \hyperlink{header_8h_a6209c6f0cbf77a7146be05f6df115ceb}{stck} $\ast$\hyperlink{structstack}{stack})
\item 
void \hyperlink{parte2_8c_a8cb17ff3dcc86bddd0bf921ae40e6fd0}{cmd\-\_\-\-D} (\hyperlink{header_8h_a341ac1667f3dc23635b071398592e724}{E\-S\-T\-R\-U\-T\-U\-R\-A\-\_\-\-B\-N} $\ast$\hyperlink{structest__bn}{est\-\_\-bn}, \hyperlink{header_8h_a6209c6f0cbf77a7146be05f6df115ceb}{stck} $\ast$\hyperlink{structstack}{stack})
\item 
void \hyperlink{parte2_8c_ae761aedc00080a6f00d322dcc397b1e9}{push} (\hyperlink{header_8h_a6209c6f0cbf77a7146be05f6df115ceb}{stck} $\ast$\hyperlink{structstack}{stack}, char cmd, union dados d)
\item 
void \hyperlink{parte2_8c_abfd19c9e6e4452b9d482ee7b8a80bcf8}{pop} (\hyperlink{header_8h_a6209c6f0cbf77a7146be05f6df115ceb}{stck} $\ast$\hyperlink{structstack}{stack})
\item 
int \hyperlink{parte2_8c_a264ed0392b7344640b191d8859dee860}{cmd\-\_\-l} (\hyperlink{header_8h_a341ac1667f3dc23635b071398592e724}{E\-S\-T\-R\-U\-T\-U\-R\-A\-\_\-\-B\-N} $\ast$\hyperlink{structest__bn}{est\-\_\-bn}, char $\ast$path, \hyperlink{header_8h_a6209c6f0cbf77a7146be05f6df115ceb}{stck} $\ast$\hyperlink{structstack}{stack})
\item 
int \hyperlink{parte2_8c_a5cd6b4b254cce7dd3ac53266652364f3}{cmd\-\_\-e} (\hyperlink{header_8h_a341ac1667f3dc23635b071398592e724}{E\-S\-T\-R\-U\-T\-U\-R\-A\-\_\-\-B\-N} $\ast$\hyperlink{structest__bn}{est\-\_\-bn}, char $\ast$path)
\item 
int \hyperlink{parte2_8c_a4ccee3cf1072baccbb5b7dc811af1d10}{verifica\-Agua} (\hyperlink{header_8h_a341ac1667f3dc23635b071398592e724}{E\-S\-T\-R\-U\-T\-U\-R\-A\-\_\-\-B\-N} $\ast$\hyperlink{structest__bn}{est\-\_\-bn})
\item 
int \hyperlink{parte2_8c_a1664d23036bf45ce3785feee253d047c}{verifica\-Cantos\-P} (\hyperlink{header_8h_a341ac1667f3dc23635b071398592e724}{E\-S\-T\-R\-U\-T\-U\-R\-A\-\_\-\-B\-N} $\ast$\hyperlink{structest__bn}{est\-\_\-bn})
\item 
int \hyperlink{parte2_8c_a794c7feec2df77f7f3c979880605f9a4}{verifica\-S\-U\-B} (\hyperlink{header_8h_a341ac1667f3dc23635b071398592e724}{E\-S\-T\-R\-U\-T\-U\-R\-A\-\_\-\-B\-N} $\ast$\hyperlink{structest__bn}{est\-\_\-bn}, char peca, int lin, int col)
\item 
int \hyperlink{parte2_8c_a8d4794f085526f1dfdb447e0c1ccfc27}{verifica\-Barcos} (\hyperlink{header_8h_a341ac1667f3dc23635b071398592e724}{E\-S\-T\-R\-U\-T\-U\-R\-A\-\_\-\-B\-N} $\ast$\hyperlink{structest__bn}{est\-\_\-bn})
\item 
int \hyperlink{parte2_8c_abb8e46fa52eb33ebc8727cd91d5d3d43}{verifica\-Peca\-Ex} (\hyperlink{header_8h_a341ac1667f3dc23635b071398592e724}{E\-S\-T\-R\-U\-T\-U\-R\-A\-\_\-\-B\-N} $\ast$\hyperlink{structest__bn}{est\-\_\-bn})
\item 
int \hyperlink{parte2_8c_ab447d04cb4b32bf8329f6016ef4324a8}{verifica\-Seg} (\hyperlink{header_8h_a341ac1667f3dc23635b071398592e724}{E\-S\-T\-R\-U\-T\-U\-R\-A\-\_\-\-B\-N} $\ast$\hyperlink{structest__bn}{est\-\_\-bn})
\item 
int \hyperlink{parte2_8c_ade8970e1ef02b807acf8ac6ee37d68ec}{cmd\-\_\-\-V} (\hyperlink{header_8h_a341ac1667f3dc23635b071398592e724}{E\-S\-T\-R\-U\-T\-U\-R\-A\-\_\-\-B\-N} $\ast$\hyperlink{structest__bn}{est\-\_\-bn})
\item 
void \hyperlink{parte2_8c_a0e42340b81e723cfce403ee87e801a63}{E1\-\_\-primlinha} (\hyperlink{header_8h_a341ac1667f3dc23635b071398592e724}{E\-S\-T\-R\-U\-T\-U\-R\-A\-\_\-\-B\-N} $\ast$\hyperlink{structest__bn}{est\-\_\-bn})
\item 
void \hyperlink{parte2_8c_a0e4de54db8be6b54c41c64766e6c2add}{E1\-\_\-filtraresto} (\hyperlink{header_8h_a341ac1667f3dc23635b071398592e724}{E\-S\-T\-R\-U\-T\-U\-R\-A\-\_\-\-B\-N} $\ast$\hyperlink{structest__bn}{est\-\_\-bn})
\item 
int \hyperlink{parte2_8c_a532bbe304aed8c44bb06e519d4cfb30e}{contaaguas} (\hyperlink{header_8h_a341ac1667f3dc23635b071398592e724}{E\-S\-T\-R\-U\-T\-U\-R\-A\-\_\-\-B\-N} $\ast$\hyperlink{structest__bn}{est\-\_\-bn})
\item 
int \hyperlink{parte2_8c_aeda5c885aef71ee4ce5dcb11d3dcf62d}{cmd\-\_\-\-E1} (\hyperlink{header_8h_a341ac1667f3dc23635b071398592e724}{E\-S\-T\-R\-U\-T\-U\-R\-A\-\_\-\-B\-N} $\ast$\hyperlink{structest__bn}{est\-\_\-bn}, \hyperlink{header_8h_a6209c6f0cbf77a7146be05f6df115ceb}{stck} $\ast$\hyperlink{structstack}{stack})
\item 
int \hyperlink{parte2_8c_ad327b2dee70f4002d5f94cf71b076095}{pertence} (char c)
\item 
void \hyperlink{parte2_8c_a4893d717483b7e828a692e9b49371953}{vaux} (\hyperlink{header_8h_a341ac1667f3dc23635b071398592e724}{E\-S\-T\-R\-U\-T\-U\-R\-A\-\_\-\-B\-N} $\ast$\hyperlink{structest__bn}{est\-\_\-bn}, int c)
\item 
int \hyperlink{parte2_8c_af373aa1eabe08a0bc3525b40acc2166f}{E2\-C} (\hyperlink{header_8h_a341ac1667f3dc23635b071398592e724}{E\-S\-T\-R\-U\-T\-U\-R\-A\-\_\-\-B\-N} $\ast$\hyperlink{structest__bn}{est\-\_\-bn})
\item 
void \hyperlink{parte2_8c_ad82ef6266d363ce176f2785e1525f736}{haux} (\hyperlink{header_8h_a341ac1667f3dc23635b071398592e724}{E\-S\-T\-R\-U\-T\-U\-R\-A\-\_\-\-B\-N} $\ast$\hyperlink{structest__bn}{est\-\_\-bn}, int l)
\item 
int \hyperlink{parte2_8c_a737aaaf90d4c029e43ae5c7197c368b7}{E2\-L} (\hyperlink{header_8h_a341ac1667f3dc23635b071398592e724}{E\-S\-T\-R\-U\-T\-U\-R\-A\-\_\-\-B\-N} $\ast$\hyperlink{structest__bn}{est\-\_\-bn})
\item 
int \hyperlink{parte2_8c_ac1f8bee34a3b5ea23735a6d86a21c4e1}{cmd\-\_\-\-E2} (\hyperlink{header_8h_a341ac1667f3dc23635b071398592e724}{E\-S\-T\-R\-U\-T\-U\-R\-A\-\_\-\-B\-N} $\ast$\hyperlink{structest__bn}{est\-\_\-bn}, \hyperlink{header_8h_a6209c6f0cbf77a7146be05f6df115ceb}{stck} $\ast$\hyperlink{structstack}{stack})
\item 
int \hyperlink{parte2_8c_a2502b7e46b4c851ba37d29d00f606167}{pertence3} (char c)
\item 
int \hyperlink{parte2_8c_aff1776159d9e217fd3a8abefdabafce1}{contasegl} (\hyperlink{header_8h_a341ac1667f3dc23635b071398592e724}{E\-S\-T\-R\-U\-T\-U\-R\-A\-\_\-\-B\-N} $\ast$\hyperlink{structest__bn}{est\-\_\-bn}, int l)
\item 
int \hyperlink{parte2_8c_a02bc4d36c6020cfc3e680115457ed6e0}{contapl} (\hyperlink{header_8h_a341ac1667f3dc23635b071398592e724}{E\-S\-T\-R\-U\-T\-U\-R\-A\-\_\-\-B\-N} $\ast$\hyperlink{structest__bn}{est\-\_\-bn}, int l)
\item 
void \hyperlink{parte2_8c_a8e4e2797850fab71b598360977d7c757}{porosl} (\hyperlink{header_8h_a341ac1667f3dc23635b071398592e724}{E\-S\-T\-R\-U\-T\-U\-R\-A\-\_\-\-B\-N} $\ast$\hyperlink{structest__bn}{est\-\_\-bn})
\item 
int \hyperlink{parte2_8c_a9f91b9e2ea6340139226cbf13b56377c}{contasegc} (\hyperlink{header_8h_a341ac1667f3dc23635b071398592e724}{E\-S\-T\-R\-U\-T\-U\-R\-A\-\_\-\-B\-N} $\ast$\hyperlink{structest__bn}{est\-\_\-bn}, int c)
\item 
int \hyperlink{parte2_8c_a257e5c96ec209788e5ba64374b02b448}{contapc} (\hyperlink{header_8h_a341ac1667f3dc23635b071398592e724}{E\-S\-T\-R\-U\-T\-U\-R\-A\-\_\-\-B\-N} $\ast$\hyperlink{structest__bn}{est\-\_\-bn}, int c)
\item 
void \hyperlink{parte2_8c_a518348f7e391149236896ec636e9a00f}{porosc} (\hyperlink{header_8h_a341ac1667f3dc23635b071398592e724}{E\-S\-T\-R\-U\-T\-U\-R\-A\-\_\-\-B\-N} $\ast$\hyperlink{structest__bn}{est\-\_\-bn})
\item 
void \hyperlink{parte2_8c_ae9becda6b15e9175ac1da8554f5cbcac}{poros} (\hyperlink{header_8h_a341ac1667f3dc23635b071398592e724}{E\-S\-T\-R\-U\-T\-U\-R\-A\-\_\-\-B\-N} $\ast$\hyperlink{structest__bn}{est\-\_\-bn})
\item 
int \hyperlink{parte2_8c_a3b61ee2b26716e43a993ce9cc4d3a0d5}{porsubs} (\hyperlink{header_8h_a341ac1667f3dc23635b071398592e724}{E\-S\-T\-R\-U\-T\-U\-R\-A\-\_\-\-B\-N} $\ast$\hyperlink{structest__bn}{est\-\_\-bn})
\item 
void \hyperlink{parte2_8c_a3be1b4c4216fd6ce515499ab4d1645b4}{poebarcos\-Auxl} (\hyperlink{header_8h_a341ac1667f3dc23635b071398592e724}{E\-S\-T\-R\-U\-T\-U\-R\-A\-\_\-\-B\-N} $\ast$\hyperlink{structest__bn}{est\-\_\-bn}, int l, int c)
\item 
void \hyperlink{parte2_8c_a80813024d0bc6109d936ab88048dbd6e}{caso\-El} (\hyperlink{header_8h_a341ac1667f3dc23635b071398592e724}{E\-S\-T\-R\-U\-T\-U\-R\-A\-\_\-\-B\-N} $\ast$\hyperlink{structest__bn}{est\-\_\-bn}, int l, int c)
\item 
void \hyperlink{parte2_8c_af633d2c99bab94c2496e33521ecc6455}{caso\-Ec} (\hyperlink{header_8h_a341ac1667f3dc23635b071398592e724}{E\-S\-T\-R\-U\-T\-U\-R\-A\-\_\-\-B\-N} $\ast$\hyperlink{structest__bn}{est\-\_\-bn}, int l, int c)
\item 
void \hyperlink{parte2_8c_ac8f5860cb4be216d6d43fd0de19b8813}{poebarcos\-Auxc} (\hyperlink{header_8h_a341ac1667f3dc23635b071398592e724}{E\-S\-T\-R\-U\-T\-U\-R\-A\-\_\-\-B\-N} $\ast$\hyperlink{structest__bn}{est\-\_\-bn}, int l, int c)
\item 
int \hyperlink{parte2_8c_a24d71c1ffe070ee60f929e659b709130}{porbarcos} (\hyperlink{header_8h_a341ac1667f3dc23635b071398592e724}{E\-S\-T\-R\-U\-T\-U\-R\-A\-\_\-\-B\-N} $\ast$\hyperlink{structest__bn}{est\-\_\-bn})
\item 
int \hyperlink{parte2_8c_a436309221725e30fde866af67569beba}{porsegs} (\hyperlink{header_8h_a341ac1667f3dc23635b071398592e724}{E\-S\-T\-R\-U\-T\-U\-R\-A\-\_\-\-B\-N} $\ast$\hyperlink{structest__bn}{est\-\_\-bn})
\item 
void \hyperlink{parte2_8c_a3430f05726117d6094e9236eb96f4d93}{tiraros} (\hyperlink{header_8h_a341ac1667f3dc23635b071398592e724}{E\-S\-T\-R\-U\-T\-U\-R\-A\-\_\-\-B\-N} $\ast$\hyperlink{structest__bn}{est\-\_\-bn})
\item 
int \hyperlink{parte2_8c_a5b883af808e6d414a5ee0d2623ad3a94}{cmd\-\_\-\-E3} (\hyperlink{header_8h_a341ac1667f3dc23635b071398592e724}{E\-S\-T\-R\-U\-T\-U\-R\-A\-\_\-\-B\-N} $\ast$\hyperlink{structest__bn}{est\-\_\-bn}, \hyperlink{header_8h_a6209c6f0cbf77a7146be05f6df115ceb}{stck} $\ast$\hyperlink{structstack}{stack})
\end{DoxyCompactItemize}


\subsection{Documentação das funções}
\hypertarget{parte2_8c_af633d2c99bab94c2496e33521ecc6455}{\index{parte2.\-c@{parte2.\-c}!caso\-Ec@{caso\-Ec}}
\index{caso\-Ec@{caso\-Ec}!parte2.c@{parte2.\-c}}
\subsubsection[{caso\-Ec}]{\setlength{\rightskip}{0pt plus 5cm}void caso\-Ec (
\begin{DoxyParamCaption}
\item[{{\bf E\-S\-T\-R\-U\-T\-U\-R\-A\-\_\-\-B\-N} $\ast$}]{est\-\_\-bn, }
\item[{int}]{l, }
\item[{int}]{c}
\end{DoxyParamCaption}
)}}\label{parte2_8c_af633d2c99bab94c2496e33521ecc6455}
\hypertarget{parte2_8c_a80813024d0bc6109d936ab88048dbd6e}{\index{parte2.\-c@{parte2.\-c}!caso\-El@{caso\-El}}
\index{caso\-El@{caso\-El}!parte2.c@{parte2.\-c}}
\subsubsection[{caso\-El}]{\setlength{\rightskip}{0pt plus 5cm}void caso\-El (
\begin{DoxyParamCaption}
\item[{{\bf E\-S\-T\-R\-U\-T\-U\-R\-A\-\_\-\-B\-N} $\ast$}]{est\-\_\-bn, }
\item[{int}]{l, }
\item[{int}]{c}
\end{DoxyParamCaption}
)}}\label{parte2_8c_a80813024d0bc6109d936ab88048dbd6e}
\hypertarget{parte2_8c_a8cb17ff3dcc86bddd0bf921ae40e6fd0}{\index{parte2.\-c@{parte2.\-c}!cmd\-\_\-\-D@{cmd\-\_\-\-D}}
\index{cmd\-\_\-\-D@{cmd\-\_\-\-D}!parte2.c@{parte2.\-c}}
\subsubsection[{cmd\-\_\-\-D}]{\setlength{\rightskip}{0pt plus 5cm}void cmd\-\_\-\-D (
\begin{DoxyParamCaption}
\item[{{\bf E\-S\-T\-R\-U\-T\-U\-R\-A\-\_\-\-B\-N} $\ast$}]{est\-\_\-bn, }
\item[{{\bf stck} $\ast$}]{stack}
\end{DoxyParamCaption}
)}}\label{parte2_8c_a8cb17ff3dcc86bddd0bf921ae40e6fd0}
O comando D desfaz um comando e volta ao estado anterior.


\begin{DoxyParams}{Parâmetros}
{\em A} & função desfaz a última alteração feita ao tabuleiro. \\
\hline
{\em Esta} & é constituida por várias funções auxiliares pois cada comando é diferente na sua forma de alterar o tabuleiro.\\
\hline
\end{DoxyParams}
\begin{DoxyReturn}{Retorna}
Não retorna nada apenas desfaz a última alteração feita ao tabuleiro. 
\end{DoxyReturn}
\hypertarget{parte2_8c_a5cd6b4b254cce7dd3ac53266652364f3}{\index{parte2.\-c@{parte2.\-c}!cmd\-\_\-e@{cmd\-\_\-e}}
\index{cmd\-\_\-e@{cmd\-\_\-e}!parte2.c@{parte2.\-c}}
\subsubsection[{cmd\-\_\-e}]{\setlength{\rightskip}{0pt plus 5cm}int cmd\-\_\-e (
\begin{DoxyParamCaption}
\item[{{\bf E\-S\-T\-R\-U\-T\-U\-R\-A\-\_\-\-B\-N} $\ast$}]{est\-\_\-bn, }
\item[{char $\ast$}]{path}
\end{DoxyParamCaption}
)}}\label{parte2_8c_a5cd6b4b254cce7dd3ac53266652364f3}
O comando e escrever um tabuleiro num ficheiro .txt.


\begin{DoxyParams}{Parâmetros}
{\em A} & função vai receber um tabuleiro a partir do qual vai escrever num dado ficheiro, para o qual teremos de escolher o nome e o path.\\
\hline
\end{DoxyParams}
\begin{DoxyReturn}{Retorna}
Não retorna nada apenas escreve no ficheiro .txt. 
\end{DoxyReturn}
\hypertarget{parte2_8c_aeda5c885aef71ee4ce5dcb11d3dcf62d}{\index{parte2.\-c@{parte2.\-c}!cmd\-\_\-\-E1@{cmd\-\_\-\-E1}}
\index{cmd\-\_\-\-E1@{cmd\-\_\-\-E1}!parte2.c@{parte2.\-c}}
\subsubsection[{cmd\-\_\-\-E1}]{\setlength{\rightskip}{0pt plus 5cm}int cmd\-\_\-\-E1 (
\begin{DoxyParamCaption}
\item[{{\bf E\-S\-T\-R\-U\-T\-U\-R\-A\-\_\-\-B\-N} $\ast$}]{est\-\_\-bn, }
\item[{{\bf stck} $\ast$}]{stack}
\end{DoxyParamCaption}
)}}\label{parte2_8c_aeda5c885aef71ee4ce5dcb11d3dcf62d}
O comando E1 é a estratégia 1.


\begin{DoxyParams}{Parâmetros}
{\em A} & função recebe o tabuleiro e é responsável por colocar água onde se deduz que vai existir à volta de todos os segmentos de barcos já colocados. \\
\hline
{\em É} & constituída por duas funções auxiliares chamadas E1\-\_\-primlinha e E1\-\_\-filtraresto que começam por preencher a primeira linha. Caso existam mais do que uma linha, esta preenchê-\/las-\/á.\\
\hline
\end{DoxyParams}
\begin{DoxyReturn}{Retorna}
Retorna 1 se houverem alterações e 0 se não houverem. 
\end{DoxyReturn}
\hypertarget{parte2_8c_ac1f8bee34a3b5ea23735a6d86a21c4e1}{\index{parte2.\-c@{parte2.\-c}!cmd\-\_\-\-E2@{cmd\-\_\-\-E2}}
\index{cmd\-\_\-\-E2@{cmd\-\_\-\-E2}!parte2.c@{parte2.\-c}}
\subsubsection[{cmd\-\_\-\-E2}]{\setlength{\rightskip}{0pt plus 5cm}int cmd\-\_\-\-E2 (
\begin{DoxyParamCaption}
\item[{{\bf E\-S\-T\-R\-U\-T\-U\-R\-A\-\_\-\-B\-N} $\ast$}]{est\-\_\-bn, }
\item[{{\bf stck} $\ast$}]{stack}
\end{DoxyParamCaption}
)}}\label{parte2_8c_ac1f8bee34a3b5ea23735a6d86a21c4e1}
O comando E2 é a estratégia 2.


\begin{DoxyParams}{Parâmetros}
{\em A} & função recebe o tabuleiro e é responsável por colocar água nas linhas e colunas em todos os segmentos de barcos que já foram colocados.\\
\hline
\end{DoxyParams}
\begin{DoxyReturn}{Retorna}
retorna 1 se houverem alterações e 0 se não houverem. 
\end{DoxyReturn}
\hypertarget{parte2_8c_a5b883af808e6d414a5ee0d2623ad3a94}{\index{parte2.\-c@{parte2.\-c}!cmd\-\_\-\-E3@{cmd\-\_\-\-E3}}
\index{cmd\-\_\-\-E3@{cmd\-\_\-\-E3}!parte2.c@{parte2.\-c}}
\subsubsection[{cmd\-\_\-\-E3}]{\setlength{\rightskip}{0pt plus 5cm}int cmd\-\_\-\-E3 (
\begin{DoxyParamCaption}
\item[{{\bf E\-S\-T\-R\-U\-T\-U\-R\-A\-\_\-\-B\-N} $\ast$}]{est\-\_\-bn, }
\item[{{\bf stck} $\ast$}]{stack}
\end{DoxyParamCaption}
)}}\label{parte2_8c_a5b883af808e6d414a5ee0d2623ad3a94}
O comando E3 é a estratégia E3


\begin{DoxyParams}{Parâmetros}
{\em A} & função recebe um tabuleiro quase completo, ou seja onde a água já esteja completamente preenchida, e coloca segmentos de barcos nas linhas e colunas nas quais todos os espaços vazios têm que conter segmentos de barcos para que o número correspondente seja respeitado.\\
\hline
\end{DoxyParams}
\begin{DoxyReturn}{Retorna}
A funcao retorna 1 se houverem alterações e 0 se não houverem. 
\end{DoxyReturn}
\hypertarget{parte2_8c_a264ed0392b7344640b191d8859dee860}{\index{parte2.\-c@{parte2.\-c}!cmd\-\_\-l@{cmd\-\_\-l}}
\index{cmd\-\_\-l@{cmd\-\_\-l}!parte2.c@{parte2.\-c}}
\subsubsection[{cmd\-\_\-l}]{\setlength{\rightskip}{0pt plus 5cm}int cmd\-\_\-l (
\begin{DoxyParamCaption}
\item[{{\bf E\-S\-T\-R\-U\-T\-U\-R\-A\-\_\-\-B\-N} $\ast$}]{est\-\_\-bn, }
\item[{char $\ast$}]{path, }
\item[{{\bf stck} $\ast$}]{stack}
\end{DoxyParamCaption}
)}}\label{parte2_8c_a264ed0392b7344640b191d8859dee860}
O comando l lê o tabuleiro a partir de um ficheiro .txt.


\begin{DoxyParams}{Parâmetros}
{\em A} & função recebe um tabuleiro a partir do ficheiro .txt escrito que pertence à mesma diretoria onde o projeto se encontra.\\
\hline
\end{DoxyParams}
\begin{DoxyReturn}{Retorna}
Não retorna nada apenas lê o tabuleiro. 
\end{DoxyReturn}
\hypertarget{parte2_8c_ade8970e1ef02b807acf8ac6ee37d68ec}{\index{parte2.\-c@{parte2.\-c}!cmd\-\_\-\-V@{cmd\-\_\-\-V}}
\index{cmd\-\_\-\-V@{cmd\-\_\-\-V}!parte2.c@{parte2.\-c}}
\subsubsection[{cmd\-\_\-\-V}]{\setlength{\rightskip}{0pt plus 5cm}int cmd\-\_\-\-V (
\begin{DoxyParamCaption}
\item[{{\bf E\-S\-T\-R\-U\-T\-U\-R\-A\-\_\-\-B\-N} $\ast$}]{est\-\_\-bn}
\end{DoxyParamCaption}
)}}\label{parte2_8c_ade8970e1ef02b807acf8ac6ee37d68ec}
O comando V verifica se a solução apresentada está correta.


\begin{DoxyParams}{Parâmetros}
{\em A} & função é composta por várias funções auxiliares que conforme as regras da água, dos segmentos e dos barcos verifica se a solução apresentada está correta\\
\hline
\end{DoxyParams}
\begin{DoxyReturn}{Retorna}
A função retorna S\-I\-M caso a solução seja a correta e retorna N\-A\-O caso seja incorreta. 
\end{DoxyReturn}
\hypertarget{parte2_8c_a532bbe304aed8c44bb06e519d4cfb30e}{\index{parte2.\-c@{parte2.\-c}!contaaguas@{contaaguas}}
\index{contaaguas@{contaaguas}!parte2.c@{parte2.\-c}}
\subsubsection[{contaaguas}]{\setlength{\rightskip}{0pt plus 5cm}int contaaguas (
\begin{DoxyParamCaption}
\item[{{\bf E\-S\-T\-R\-U\-T\-U\-R\-A\-\_\-\-B\-N} $\ast$}]{est\-\_\-bn}
\end{DoxyParamCaption}
)}}\label{parte2_8c_a532bbe304aed8c44bb06e519d4cfb30e}
\hypertarget{parte2_8c_a257e5c96ec209788e5ba64374b02b448}{\index{parte2.\-c@{parte2.\-c}!contapc@{contapc}}
\index{contapc@{contapc}!parte2.c@{parte2.\-c}}
\subsubsection[{contapc}]{\setlength{\rightskip}{0pt plus 5cm}int contapc (
\begin{DoxyParamCaption}
\item[{{\bf E\-S\-T\-R\-U\-T\-U\-R\-A\-\_\-\-B\-N} $\ast$}]{est\-\_\-bn, }
\item[{int}]{c}
\end{DoxyParamCaption}
)}}\label{parte2_8c_a257e5c96ec209788e5ba64374b02b448}
\hypertarget{parte2_8c_a02bc4d36c6020cfc3e680115457ed6e0}{\index{parte2.\-c@{parte2.\-c}!contapl@{contapl}}
\index{contapl@{contapl}!parte2.c@{parte2.\-c}}
\subsubsection[{contapl}]{\setlength{\rightskip}{0pt plus 5cm}int contapl (
\begin{DoxyParamCaption}
\item[{{\bf E\-S\-T\-R\-U\-T\-U\-R\-A\-\_\-\-B\-N} $\ast$}]{est\-\_\-bn, }
\item[{int}]{l}
\end{DoxyParamCaption}
)}}\label{parte2_8c_a02bc4d36c6020cfc3e680115457ed6e0}
\hypertarget{parte2_8c_a9f91b9e2ea6340139226cbf13b56377c}{\index{parte2.\-c@{parte2.\-c}!contasegc@{contasegc}}
\index{contasegc@{contasegc}!parte2.c@{parte2.\-c}}
\subsubsection[{contasegc}]{\setlength{\rightskip}{0pt plus 5cm}int contasegc (
\begin{DoxyParamCaption}
\item[{{\bf E\-S\-T\-R\-U\-T\-U\-R\-A\-\_\-\-B\-N} $\ast$}]{est\-\_\-bn, }
\item[{int}]{c}
\end{DoxyParamCaption}
)}}\label{parte2_8c_a9f91b9e2ea6340139226cbf13b56377c}
\hypertarget{parte2_8c_aff1776159d9e217fd3a8abefdabafce1}{\index{parte2.\-c@{parte2.\-c}!contasegl@{contasegl}}
\index{contasegl@{contasegl}!parte2.c@{parte2.\-c}}
\subsubsection[{contasegl}]{\setlength{\rightskip}{0pt plus 5cm}int contasegl (
\begin{DoxyParamCaption}
\item[{{\bf E\-S\-T\-R\-U\-T\-U\-R\-A\-\_\-\-B\-N} $\ast$}]{est\-\_\-bn, }
\item[{int}]{l}
\end{DoxyParamCaption}
)}}\label{parte2_8c_aff1776159d9e217fd3a8abefdabafce1}
\hypertarget{parte2_8c_a0e4de54db8be6b54c41c64766e6c2add}{\index{parte2.\-c@{parte2.\-c}!E1\-\_\-filtraresto@{E1\-\_\-filtraresto}}
\index{E1\-\_\-filtraresto@{E1\-\_\-filtraresto}!parte2.c@{parte2.\-c}}
\subsubsection[{E1\-\_\-filtraresto}]{\setlength{\rightskip}{0pt plus 5cm}void E1\-\_\-filtraresto (
\begin{DoxyParamCaption}
\item[{{\bf E\-S\-T\-R\-U\-T\-U\-R\-A\-\_\-\-B\-N} $\ast$}]{est\-\_\-bn}
\end{DoxyParamCaption}
)}}\label{parte2_8c_a0e4de54db8be6b54c41c64766e6c2add}
\hypertarget{parte2_8c_a0e42340b81e723cfce403ee87e801a63}{\index{parte2.\-c@{parte2.\-c}!E1\-\_\-primlinha@{E1\-\_\-primlinha}}
\index{E1\-\_\-primlinha@{E1\-\_\-primlinha}!parte2.c@{parte2.\-c}}
\subsubsection[{E1\-\_\-primlinha}]{\setlength{\rightskip}{0pt plus 5cm}void E1\-\_\-primlinha (
\begin{DoxyParamCaption}
\item[{{\bf E\-S\-T\-R\-U\-T\-U\-R\-A\-\_\-\-B\-N} $\ast$}]{est\-\_\-bn}
\end{DoxyParamCaption}
)}}\label{parte2_8c_a0e42340b81e723cfce403ee87e801a63}
\hypertarget{parte2_8c_af373aa1eabe08a0bc3525b40acc2166f}{\index{parte2.\-c@{parte2.\-c}!E2\-C@{E2\-C}}
\index{E2\-C@{E2\-C}!parte2.c@{parte2.\-c}}
\subsubsection[{E2\-C}]{\setlength{\rightskip}{0pt plus 5cm}int E2\-C (
\begin{DoxyParamCaption}
\item[{{\bf E\-S\-T\-R\-U\-T\-U\-R\-A\-\_\-\-B\-N} $\ast$}]{est\-\_\-bn}
\end{DoxyParamCaption}
)}}\label{parte2_8c_af373aa1eabe08a0bc3525b40acc2166f}
\hypertarget{parte2_8c_a737aaaf90d4c029e43ae5c7197c368b7}{\index{parte2.\-c@{parte2.\-c}!E2\-L@{E2\-L}}
\index{E2\-L@{E2\-L}!parte2.c@{parte2.\-c}}
\subsubsection[{E2\-L}]{\setlength{\rightskip}{0pt plus 5cm}int E2\-L (
\begin{DoxyParamCaption}
\item[{{\bf E\-S\-T\-R\-U\-T\-U\-R\-A\-\_\-\-B\-N} $\ast$}]{est\-\_\-bn}
\end{DoxyParamCaption}
)}}\label{parte2_8c_a737aaaf90d4c029e43ae5c7197c368b7}
\hypertarget{parte2_8c_ad82ef6266d363ce176f2785e1525f736}{\index{parte2.\-c@{parte2.\-c}!haux@{haux}}
\index{haux@{haux}!parte2.c@{parte2.\-c}}
\subsubsection[{haux}]{\setlength{\rightskip}{0pt plus 5cm}void haux (
\begin{DoxyParamCaption}
\item[{{\bf E\-S\-T\-R\-U\-T\-U\-R\-A\-\_\-\-B\-N} $\ast$}]{est\-\_\-bn, }
\item[{int}]{l}
\end{DoxyParamCaption}
)}}\label{parte2_8c_ad82ef6266d363ce176f2785e1525f736}
\hypertarget{parte2_8c_ad327b2dee70f4002d5f94cf71b076095}{\index{parte2.\-c@{parte2.\-c}!pertence@{pertence}}
\index{pertence@{pertence}!parte2.c@{parte2.\-c}}
\subsubsection[{pertence}]{\setlength{\rightskip}{0pt plus 5cm}int pertence (
\begin{DoxyParamCaption}
\item[{char}]{c}
\end{DoxyParamCaption}
)}}\label{parte2_8c_ad327b2dee70f4002d5f94cf71b076095}
\hypertarget{parte2_8c_a2502b7e46b4c851ba37d29d00f606167}{\index{parte2.\-c@{parte2.\-c}!pertence3@{pertence3}}
\index{pertence3@{pertence3}!parte2.c@{parte2.\-c}}
\subsubsection[{pertence3}]{\setlength{\rightskip}{0pt plus 5cm}int pertence3 (
\begin{DoxyParamCaption}
\item[{char}]{c}
\end{DoxyParamCaption}
)}}\label{parte2_8c_a2502b7e46b4c851ba37d29d00f606167}
\hypertarget{parte2_8c_ac8f5860cb4be216d6d43fd0de19b8813}{\index{parte2.\-c@{parte2.\-c}!poebarcos\-Auxc@{poebarcos\-Auxc}}
\index{poebarcos\-Auxc@{poebarcos\-Auxc}!parte2.c@{parte2.\-c}}
\subsubsection[{poebarcos\-Auxc}]{\setlength{\rightskip}{0pt plus 5cm}void poebarcos\-Auxc (
\begin{DoxyParamCaption}
\item[{{\bf E\-S\-T\-R\-U\-T\-U\-R\-A\-\_\-\-B\-N} $\ast$}]{est\-\_\-bn, }
\item[{int}]{l, }
\item[{int}]{c}
\end{DoxyParamCaption}
)}}\label{parte2_8c_ac8f5860cb4be216d6d43fd0de19b8813}
\hypertarget{parte2_8c_a3be1b4c4216fd6ce515499ab4d1645b4}{\index{parte2.\-c@{parte2.\-c}!poebarcos\-Auxl@{poebarcos\-Auxl}}
\index{poebarcos\-Auxl@{poebarcos\-Auxl}!parte2.c@{parte2.\-c}}
\subsubsection[{poebarcos\-Auxl}]{\setlength{\rightskip}{0pt plus 5cm}void poebarcos\-Auxl (
\begin{DoxyParamCaption}
\item[{{\bf E\-S\-T\-R\-U\-T\-U\-R\-A\-\_\-\-B\-N} $\ast$}]{est\-\_\-bn, }
\item[{int}]{l, }
\item[{int}]{c}
\end{DoxyParamCaption}
)}}\label{parte2_8c_a3be1b4c4216fd6ce515499ab4d1645b4}
\hypertarget{parte2_8c_abfd19c9e6e4452b9d482ee7b8a80bcf8}{\index{parte2.\-c@{parte2.\-c}!pop@{pop}}
\index{pop@{pop}!parte2.c@{parte2.\-c}}
\subsubsection[{pop}]{\setlength{\rightskip}{0pt plus 5cm}void pop (
\begin{DoxyParamCaption}
\item[{{\bf stck} $\ast$}]{stack}
\end{DoxyParamCaption}
)}}\label{parte2_8c_abfd19c9e6e4452b9d482ee7b8a80bcf8}
Esta função retira um elemento da stack.


\begin{DoxyParams}{Parâmetros}
{\em esta} & função recebe somente a stack. \\
\hline
\end{DoxyParams}
\hypertarget{parte2_8c_a24d71c1ffe070ee60f929e659b709130}{\index{parte2.\-c@{parte2.\-c}!porbarcos@{porbarcos}}
\index{porbarcos@{porbarcos}!parte2.c@{parte2.\-c}}
\subsubsection[{porbarcos}]{\setlength{\rightskip}{0pt plus 5cm}int porbarcos (
\begin{DoxyParamCaption}
\item[{{\bf E\-S\-T\-R\-U\-T\-U\-R\-A\-\_\-\-B\-N} $\ast$}]{est\-\_\-bn}
\end{DoxyParamCaption}
)}}\label{parte2_8c_a24d71c1ffe070ee60f929e659b709130}
\hypertarget{parte2_8c_ae9becda6b15e9175ac1da8554f5cbcac}{\index{parte2.\-c@{parte2.\-c}!poros@{poros}}
\index{poros@{poros}!parte2.c@{parte2.\-c}}
\subsubsection[{poros}]{\setlength{\rightskip}{0pt plus 5cm}void poros (
\begin{DoxyParamCaption}
\item[{{\bf E\-S\-T\-R\-U\-T\-U\-R\-A\-\_\-\-B\-N} $\ast$}]{est\-\_\-bn}
\end{DoxyParamCaption}
)}}\label{parte2_8c_ae9becda6b15e9175ac1da8554f5cbcac}
\hypertarget{parte2_8c_a518348f7e391149236896ec636e9a00f}{\index{parte2.\-c@{parte2.\-c}!porosc@{porosc}}
\index{porosc@{porosc}!parte2.c@{parte2.\-c}}
\subsubsection[{porosc}]{\setlength{\rightskip}{0pt plus 5cm}void porosc (
\begin{DoxyParamCaption}
\item[{{\bf E\-S\-T\-R\-U\-T\-U\-R\-A\-\_\-\-B\-N} $\ast$}]{est\-\_\-bn}
\end{DoxyParamCaption}
)}}\label{parte2_8c_a518348f7e391149236896ec636e9a00f}
\hypertarget{parte2_8c_a8e4e2797850fab71b598360977d7c757}{\index{parte2.\-c@{parte2.\-c}!porosl@{porosl}}
\index{porosl@{porosl}!parte2.c@{parte2.\-c}}
\subsubsection[{porosl}]{\setlength{\rightskip}{0pt plus 5cm}void porosl (
\begin{DoxyParamCaption}
\item[{{\bf E\-S\-T\-R\-U\-T\-U\-R\-A\-\_\-\-B\-N} $\ast$}]{est\-\_\-bn}
\end{DoxyParamCaption}
)}}\label{parte2_8c_a8e4e2797850fab71b598360977d7c757}
\hypertarget{parte2_8c_a436309221725e30fde866af67569beba}{\index{parte2.\-c@{parte2.\-c}!porsegs@{porsegs}}
\index{porsegs@{porsegs}!parte2.c@{parte2.\-c}}
\subsubsection[{porsegs}]{\setlength{\rightskip}{0pt plus 5cm}int porsegs (
\begin{DoxyParamCaption}
\item[{{\bf E\-S\-T\-R\-U\-T\-U\-R\-A\-\_\-\-B\-N} $\ast$}]{est\-\_\-bn}
\end{DoxyParamCaption}
)}}\label{parte2_8c_a436309221725e30fde866af67569beba}
Esta função substitui os o's que sejam possiveis substituir por segmentos de barcos.


\begin{DoxyParams}{Parâmetros}
{\em a} & função recebe a estrutura com todos os o's.\\
\hline
\end{DoxyParams}
\begin{DoxyReturn}{Retorna}
1 ou zero consoante haja alterações 
\end{DoxyReturn}
\hypertarget{parte2_8c_a3b61ee2b26716e43a993ce9cc4d3a0d5}{\index{parte2.\-c@{parte2.\-c}!porsubs@{porsubs}}
\index{porsubs@{porsubs}!parte2.c@{parte2.\-c}}
\subsubsection[{porsubs}]{\setlength{\rightskip}{0pt plus 5cm}int porsubs (
\begin{DoxyParamCaption}
\item[{{\bf E\-S\-T\-R\-U\-T\-U\-R\-A\-\_\-\-B\-N} $\ast$}]{est\-\_\-bn}
\end{DoxyParamCaption}
)}}\label{parte2_8c_a3b61ee2b26716e43a993ce9cc4d3a0d5}
\hypertarget{parte2_8c_ae761aedc00080a6f00d322dcc397b1e9}{\index{parte2.\-c@{parte2.\-c}!push@{push}}
\index{push@{push}!parte2.c@{parte2.\-c}}
\subsubsection[{push}]{\setlength{\rightskip}{0pt plus 5cm}void push (
\begin{DoxyParamCaption}
\item[{{\bf stck} $\ast$}]{stack, }
\item[{char}]{cmd, }
\item[{union dados}]{d}
\end{DoxyParamCaption}
)}}\label{parte2_8c_ae761aedc00080a6f00d322dcc397b1e9}
Esta funcao acrescenta um elemento à stack


\begin{DoxyParams}{Parâmetros}
{\em a} & função recebe a stack. \\
\hline
{\em recebe} & o comando que foi utilizado. \\
\hline
{\em recebe} & uma \char`\"{}union dados\char`\"{} que, dependendo do comando uzado será um apontador para tipos diferentes. \\
\hline
\end{DoxyParams}
\hypertarget{parte2_8c_a3430f05726117d6094e9236eb96f4d93}{\index{parte2.\-c@{parte2.\-c}!tiraros@{tiraros}}
\index{tiraros@{tiraros}!parte2.c@{parte2.\-c}}
\subsubsection[{tiraros}]{\setlength{\rightskip}{0pt plus 5cm}void tiraros (
\begin{DoxyParamCaption}
\item[{{\bf E\-S\-T\-R\-U\-T\-U\-R\-A\-\_\-\-B\-N} $\ast$}]{est\-\_\-bn}
\end{DoxyParamCaption}
)}}\label{parte2_8c_a3430f05726117d6094e9236eb96f4d93}
\hypertarget{parte2_8c_addea9cb6aecf35f686a0a33ff35456fa}{\index{parte2.\-c@{parte2.\-c}!undo\-\_\-\-C\-L\-E@{undo\-\_\-\-C\-L\-E}}
\index{undo\-\_\-\-C\-L\-E@{undo\-\_\-\-C\-L\-E}!parte2.c@{parte2.\-c}}
\subsubsection[{undo\-\_\-\-C\-L\-E}]{\setlength{\rightskip}{0pt plus 5cm}void undo\-\_\-\-C\-L\-E (
\begin{DoxyParamCaption}
\item[{{\bf E\-S\-T\-R\-U\-T\-U\-R\-A\-\_\-\-B\-N} $\ast$}]{est\-\_\-bn, }
\item[{{\bf stck} $\ast$}]{stack}
\end{DoxyParamCaption}
)}}\label{parte2_8c_addea9cb6aecf35f686a0a33ff35456fa}
Esta função está responsável por desfazer o último comando quando este foi o C, o L, ou qualquer das E's.


\begin{DoxyParams}{Parâmetros}
{\em a} & função recebe a estrutura atual \\
\hline
{\em para} & que possa recuperar a estrutura antiga, a função recebe também a stack. \\
\hline
\end{DoxyParams}
\hypertarget{parte2_8c_adda53bc08b6e1006a4a7168a7ac02d38}{\index{parte2.\-c@{parte2.\-c}!undo\-\_\-h@{undo\-\_\-h}}
\index{undo\-\_\-h@{undo\-\_\-h}!parte2.c@{parte2.\-c}}
\subsubsection[{undo\-\_\-h}]{\setlength{\rightskip}{0pt plus 5cm}void undo\-\_\-h (
\begin{DoxyParamCaption}
\item[{{\bf E\-S\-T\-R\-U\-T\-U\-R\-A\-\_\-\-B\-N} $\ast$}]{est\-\_\-bn, }
\item[{{\bf stck} $\ast$}]{stack}
\end{DoxyParamCaption}
)}}\label{parte2_8c_adda53bc08b6e1006a4a7168a7ac02d38}
Esta função está responsável por desfazer o último comando quando este foi o h.


\begin{DoxyParams}{Parâmetros}
{\em a} & função recebe a estrutura atual \\
\hline
{\em para} & que possa recuperar a estrutura antiga, a função recebe também a stack. \\
\hline
\end{DoxyParams}
\hypertarget{parte2_8c_a3612c739cc369c5547480cc53ca1b2fa}{\index{parte2.\-c@{parte2.\-c}!undo\-\_\-p@{undo\-\_\-p}}
\index{undo\-\_\-p@{undo\-\_\-p}!parte2.c@{parte2.\-c}}
\subsubsection[{undo\-\_\-p}]{\setlength{\rightskip}{0pt plus 5cm}void undo\-\_\-p (
\begin{DoxyParamCaption}
\item[{{\bf E\-S\-T\-R\-U\-T\-U\-R\-A\-\_\-\-B\-N} $\ast$}]{est\-\_\-bn, }
\item[{{\bf stck} $\ast$}]{stack}
\end{DoxyParamCaption}
)}}\label{parte2_8c_a3612c739cc369c5547480cc53ca1b2fa}
Esta função está responsável por desfazer o último comando quando este foi o p.


\begin{DoxyParams}{Parâmetros}
{\em a} & função recebe a estrutura atual \\
\hline
{\em para} & que possa recuperar a estrutura antiga, a função recebe também a stack. \\
\hline
\end{DoxyParams}
\hypertarget{parte2_8c_a530f2491dbaf2effb0937ca4fb964dfd}{\index{parte2.\-c@{parte2.\-c}!undo\-\_\-v@{undo\-\_\-v}}
\index{undo\-\_\-v@{undo\-\_\-v}!parte2.c@{parte2.\-c}}
\subsubsection[{undo\-\_\-v}]{\setlength{\rightskip}{0pt plus 5cm}void undo\-\_\-v (
\begin{DoxyParamCaption}
\item[{{\bf E\-S\-T\-R\-U\-T\-U\-R\-A\-\_\-\-B\-N} $\ast$}]{est\-\_\-bn, }
\item[{{\bf stck} $\ast$}]{stack}
\end{DoxyParamCaption}
)}}\label{parte2_8c_a530f2491dbaf2effb0937ca4fb964dfd}
Esta função está responsável por desfazer o último comando quando este foi o v.


\begin{DoxyParams}{Parâmetros}
{\em a} & função recebe a estrutura atual \\
\hline
{\em para} & que possa recuperar a estrutura antiga, a função recebe também a stack. \\
\hline
\end{DoxyParams}
\hypertarget{parte2_8c_a4893d717483b7e828a692e9b49371953}{\index{parte2.\-c@{parte2.\-c}!vaux@{vaux}}
\index{vaux@{vaux}!parte2.c@{parte2.\-c}}
\subsubsection[{vaux}]{\setlength{\rightskip}{0pt plus 5cm}void vaux (
\begin{DoxyParamCaption}
\item[{{\bf E\-S\-T\-R\-U\-T\-U\-R\-A\-\_\-\-B\-N} $\ast$}]{est\-\_\-bn, }
\item[{int}]{c}
\end{DoxyParamCaption}
)}}\label{parte2_8c_a4893d717483b7e828a692e9b49371953}
\hypertarget{parte2_8c_a4ccee3cf1072baccbb5b7dc811af1d10}{\index{parte2.\-c@{parte2.\-c}!verifica\-Agua@{verifica\-Agua}}
\index{verifica\-Agua@{verifica\-Agua}!parte2.c@{parte2.\-c}}
\subsubsection[{verifica\-Agua}]{\setlength{\rightskip}{0pt plus 5cm}int verifica\-Agua (
\begin{DoxyParamCaption}
\item[{{\bf E\-S\-T\-R\-U\-T\-U\-R\-A\-\_\-\-B\-N} $\ast$}]{est\-\_\-bn}
\end{DoxyParamCaption}
)}}\label{parte2_8c_a4ccee3cf1072baccbb5b7dc811af1d10}
\hypertarget{parte2_8c_a8d4794f085526f1dfdb447e0c1ccfc27}{\index{parte2.\-c@{parte2.\-c}!verifica\-Barcos@{verifica\-Barcos}}
\index{verifica\-Barcos@{verifica\-Barcos}!parte2.c@{parte2.\-c}}
\subsubsection[{verifica\-Barcos}]{\setlength{\rightskip}{0pt plus 5cm}int verifica\-Barcos (
\begin{DoxyParamCaption}
\item[{{\bf E\-S\-T\-R\-U\-T\-U\-R\-A\-\_\-\-B\-N} $\ast$}]{est\-\_\-bn}
\end{DoxyParamCaption}
)}}\label{parte2_8c_a8d4794f085526f1dfdb447e0c1ccfc27}
\hypertarget{parte2_8c_a1664d23036bf45ce3785feee253d047c}{\index{parte2.\-c@{parte2.\-c}!verifica\-Cantos\-P@{verifica\-Cantos\-P}}
\index{verifica\-Cantos\-P@{verifica\-Cantos\-P}!parte2.c@{parte2.\-c}}
\subsubsection[{verifica\-Cantos\-P}]{\setlength{\rightskip}{0pt plus 5cm}int verifica\-Cantos\-P (
\begin{DoxyParamCaption}
\item[{{\bf E\-S\-T\-R\-U\-T\-U\-R\-A\-\_\-\-B\-N} $\ast$}]{est\-\_\-bn}
\end{DoxyParamCaption}
)}}\label{parte2_8c_a1664d23036bf45ce3785feee253d047c}
\hypertarget{parte2_8c_abb8e46fa52eb33ebc8727cd91d5d3d43}{\index{parte2.\-c@{parte2.\-c}!verifica\-Peca\-Ex@{verifica\-Peca\-Ex}}
\index{verifica\-Peca\-Ex@{verifica\-Peca\-Ex}!parte2.c@{parte2.\-c}}
\subsubsection[{verifica\-Peca\-Ex}]{\setlength{\rightskip}{0pt plus 5cm}int verifica\-Peca\-Ex (
\begin{DoxyParamCaption}
\item[{{\bf E\-S\-T\-R\-U\-T\-U\-R\-A\-\_\-\-B\-N} $\ast$}]{est\-\_\-bn}
\end{DoxyParamCaption}
)}}\label{parte2_8c_abb8e46fa52eb33ebc8727cd91d5d3d43}
\hypertarget{parte2_8c_ab447d04cb4b32bf8329f6016ef4324a8}{\index{parte2.\-c@{parte2.\-c}!verifica\-Seg@{verifica\-Seg}}
\index{verifica\-Seg@{verifica\-Seg}!parte2.c@{parte2.\-c}}
\subsubsection[{verifica\-Seg}]{\setlength{\rightskip}{0pt plus 5cm}int verifica\-Seg (
\begin{DoxyParamCaption}
\item[{{\bf E\-S\-T\-R\-U\-T\-U\-R\-A\-\_\-\-B\-N} $\ast$}]{est\-\_\-bn}
\end{DoxyParamCaption}
)}}\label{parte2_8c_ab447d04cb4b32bf8329f6016ef4324a8}
\hypertarget{parte2_8c_a794c7feec2df77f7f3c979880605f9a4}{\index{parte2.\-c@{parte2.\-c}!verifica\-S\-U\-B@{verifica\-S\-U\-B}}
\index{verifica\-S\-U\-B@{verifica\-S\-U\-B}!parte2.c@{parte2.\-c}}
\subsubsection[{verifica\-S\-U\-B}]{\setlength{\rightskip}{0pt plus 5cm}int verifica\-S\-U\-B (
\begin{DoxyParamCaption}
\item[{{\bf E\-S\-T\-R\-U\-T\-U\-R\-A\-\_\-\-B\-N} $\ast$}]{est\-\_\-bn, }
\item[{char}]{peca, }
\item[{int}]{lin, }
\item[{int}]{col}
\end{DoxyParamCaption}
)}}\label{parte2_8c_a794c7feec2df77f7f3c979880605f9a4}

\hypertarget{parte3_8c}{\section{Referência ao ficheiro parte3.\-c}
\label{parte3_8c}\index{parte3.\-c@{parte3.\-c}}
}
{\ttfamily \#include $<$stdio.\-h$>$}\\*
{\ttfamily \#include $<$string.\-h$>$}\\*
{\ttfamily \#include $<$stdlib.\-h$>$}\\*
{\ttfamily \#include \char`\"{}header.\-h\char`\"{}}\\*
\subsection*{Funções}
\begin{DoxyCompactItemize}
\item 
void \hyperlink{parte3_8c_a22fcd06f69442842fa9634d4997d5fb5}{cmd\-\_\-\-R} (\hyperlink{header_8h_a341ac1667f3dc23635b071398592e724}{E\-S\-T\-R\-U\-T\-U\-R\-A\-\_\-\-B\-N} $\ast$\hyperlink{structest__bn}{est\-\_\-bn}, \hyperlink{header_8h_a6209c6f0cbf77a7146be05f6df115ceb}{stck} $\ast$\hyperlink{structstack}{stack})
\end{DoxyCompactItemize}


\subsection{Documentação das funções}
\hypertarget{parte3_8c_a22fcd06f69442842fa9634d4997d5fb5}{\index{parte3.\-c@{parte3.\-c}!cmd\-\_\-\-R@{cmd\-\_\-\-R}}
\index{cmd\-\_\-\-R@{cmd\-\_\-\-R}!parte3.c@{parte3.\-c}}
\subsubsection[{cmd\-\_\-\-R}]{\setlength{\rightskip}{0pt plus 5cm}void cmd\-\_\-\-R (
\begin{DoxyParamCaption}
\item[{{\bf E\-S\-T\-R\-U\-T\-U\-R\-A\-\_\-\-B\-N} $\ast$}]{est\-\_\-bn, }
\item[{{\bf stck} $\ast$}]{stack}
\end{DoxyParamCaption}
)}}\label{parte3_8c_a22fcd06f69442842fa9634d4997d5fb5}
Esta função foi criada com o intuito de resolver tabuleiros, apesar de a nossa nao estar complexa o suficiente para tal.


\begin{DoxyParams}{Parâmetros}
{\em a} & função recebe a estrutura, do tipo E\-S\-T\-R\-U\-T\-U\-R\-A\-\_\-\-B\-N. \\
\hline
{\em recebe} & também a stack com o intuito de ser possivel desfazer as alterações. \\
\hline
\end{DoxyParams}

%--- End generated contents ---

% Index
\newpage
\phantomsection
\addcontentsline{toc}{chapter}{Índice}
\printindex

\end{document}
